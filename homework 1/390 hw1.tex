\documentclass[10pt]{article}
\usepackage[utf8]{inputenc}

% \usepackage[margin=1in]{geometry} %One inch margins

%Makes things a little prettier
\usepackage{lmodern}%
\usepackage[T1]{fontenc}%
% \usepackage{microtype}%
\usepackage[varqu,varl]{inconsolata} % sans serif typewriter

% Bring in the math!
\usepackage{amsmath}%
\usepackage{amsthm}%
\usepackage{amssymb}%
\usepackage{amsfonts}%
\usepackage{inconsolata}%
% mleftright is helpful for making the parenthesis (and other delimiters) fit their contents height-wise.
\usepackage{mleftright}%
% The following macro puts parenthesis around the contents using mleftright and adapt the height.
\newcommand{\parof}[1]{\mleft( #1 \mright)}
% Of course you can define similar commands for other delimiters.

% To typeset algorithms I simply use numbered lists (e.g., enumerate). I use nested lists for inner clauses. The following is a prettier version of this approach that I use in my notes; it changes the font of the numbering to typewriter font (as a visual cue), normalizes some margins, and so forth. I often put the whole algorithm inside a \begin{quote}...\end{quote} to help set it apart from the text. You can use labels and references (as usual) to refer to steps.

\usepackage{enumitem}%

\input{preamble}
\input{letterfont}
\input{macros}

\fancyhead[L]{\tbo{Josh Park \\ Prof. Quanrud}}
\fancyhead[C]{\tbo{CS 390 ATA \\ Advanced Topics in Algorithms}}
\fancyhead[R]{\tbo{Spring 2025 \\ Homework 1}}

\begin{document}
\setcounter{section}{3}
\setcounter{exercise}{2}
\begin{exercise} % PROBLEM 3.3
  Let $ P $ be a set of $ n $ points in $ \bbR\sq $. Design and analyze an algorithm that computes a
  list of all pairs of points with distance within a factor of 2 of the minimum distance between all
  pairs of points.
\end{exercise}

\begin{quote}
\defalgo{find-pairs}{P}
\doc{Use the divide-and-conquer algorithm from lecture to find minimum distance in P.}.
\begin{steps}
  \item Let $ \mcM $ be an empty list
  \item Let $ \delta = \tmo{min-distance}(P) $\rcomm{$ \mcO(n\log n) $}
  \item For $ p \in P $: \rcomm{$ \mcO(n) $}
% \end{steps}
% \doc{Find all points contained within a square centered at $ p $ with side length $ 4\delta $.}
% \begin{steps} \setcounter{stepsi}{3}
% \item[]
  \begin{steps}
    \item Let $ Q = \{(x,y)\in P\ :\ \abs{p.x-x} < 2\delta \tand \abs{p.y-y} < 2\delta\} $
    \item For $ q \in Q $: \rcomm{$ \mcO(1) $}
    \begin{steps}
      \item If $ (q,p) \not\in \mcM $ AND $ \sqrt{(p.x-q.x)^2 + (p.y-q.y)^2} < 2\delta $:
      \begin{steps}
        \item Add $ (p,q) $ to $ \mcM $
      \end{steps}
    \end{steps}
  \end{steps}
  \item Return $ \mcM $
\end{steps}
\end{quote}
\begin{proof}[Proof of Correctness]
We know that the subroutine \tmo{min-distance}($ P $) will return the minimum distance between any two points in $ P $ in $ \mcO(n\log n) $ time. Then as we iterate through each point in $ P $ (taking $ \mcO(n) $ time), we create a square $ Q $ with side length $ 4\delta $ centered at $ p $.
  \begin{claim}
    $ Q $ contains at most 31 points.
  \end{claim}
  \begin{subproof}[Proof of Claim 1.]
  We know that the minimum distance between any two points in $ P $ is $ \delta $. For each $ q\in Q\cap P $, draw a ball of radius $ \delta/2 $ centered at $ q $. Then each ball $ B_p $ has area $ A_{B_p} = \pi \delta\sq /4 $ and lies in the ``padded'' square $ \hat Q $ with side length $ 5\delta $ and area $ A_{\hat Q} = 25\delta\sq $. Obviously every $ B_p $ must be disjoint, so the maximum number of $ B_p $ in $ \hat Q $ must be \begin{align*}
    \left\lfloor\frac{25\delta\sq}{\pi\delta\sq/4}\right\rfloor = \lt\lfloor\frac{100}{\pi}\rt\rfloor = \floor{31.8} = 31.
  \end{align*}
\end{subproof}
Thus there exists at most a constant number of points in $ Q $ and we can iterate through them in $ \mcO(1) $ time. As we iterate through each $ q\in Q $, we confirm that the pair $ (q,p) $ is not already in our list of pairs $ \mcM $ and whether the distance between $ p $ and $ q $ is less than $ 2\delta $. If both conditions are true, we add $ (p,q) $ to $ \mcM $. Thus \tmo{find-pairs} will comprehensively find all valid pairs in $ \mcO(n\log n) + \mcO(n)\mcO(1) = \mcO(n\log n) $ time.
\end{proof}
\pagebreak

\setcounter{exercise}{5}
\begin{exercise} % PROBLEM 3.6
  \nf{[The example]} parallel algorithm is very fast, but has a $ \mcO(\log n) $-factor more work than the obvious sequential algorithm.
  The goal of this exercise is to develop a (recursive) parallel divide-and-conquer algorithm that is just as fast --- $ \mcO(\log n) $ parallel time --- but has $ \mcO(n) $ total work.
  In addition to designing the algorithm, you should analyze the running time and the work similar to how we did [for the example].
\end{exercise}

\begin{quote}
\defalgo{parallel-sums}{A[1..n]}

\doc{Given an input array of numbers $ A[1..n] $, returns an array of the prefix sums of $ A $.}
\begin{steps}
  \item If $ n=1 $, return $ A $
  \item Let $ B $ be an empty array of size $ n/2 $
  \item In parallel for integers $ i\in [0, \frac{n}{2}-1]$: % \rcomm{$ \mcO(n) $}
  \begin{steps}
    \item $ B[i] = A[2i] + A[2i + 1] $
  \end{steps}
  \item $ B = \tmo{prefixSum}(B) $ % \rcomm{$ \mcO(\log n) $ work in $ \mcO(n) $ rounds}
  \item In parallel for integers $ i \in [0, \frac{n}{2}-1] $:
  \begin{steps}
    \item $ A[2i] = B[i] $
    \item $ A[2i+1] = B[i] + A[2i+1] $
  \end{steps}
  \item return $ A $
\end{steps}
\end{quote}

\begin{proof}[Proof of correctness.]
  WLOG, we may assume $ n $ is even.
  In the case $ n_{\neq 1} $ is odd, we simply pad $ A $ with a zero.

  The base case is trivial, if $ n=1 $ then the array is already its own prefix sum.

  We begin by summing pairs into $ B $ and then recursing on $ B $, which gives us the partial sums of pairs up to each index $ i $.
  We then distribute these sums back into $ A $, adjusting each pair by the total prefix sum of all previous pairs.
  Thus, the element $ A[2i] $ becomes the sum up to its own index, and $ A[2i+1] $ becomes the sum up to index $ 2i+1 $.
\end{proof}

\begin{proof}[Proof of complexity.]
  Let $ W(n) $ be the function representing the work complexity.
  Our algorithm is represented by the recurrence \begin{align*}
    W(n) &= W(n/2) + \mcO(n) \\
         &= W(n/4) + \mcO(n/2) + \mcO(n) \\
         &= W(n/2^k) + \sum_{j=0}^{k-1} \mcO(n/2^j).
  \end{align*}
  It is a basic fact that a decreasing geometric series is convergent, whence our recurrence solves to a final work complexity of $ W(n) = \mcO(n) $.

  Let $ T(n) $ be the function representing the time complexity. Then by running them in parallel, we can compute $ \mcO(n) $ sums in $ O(1) $ time. Thus our algorithm is represented by the recurrence \begin{align*}
    T(n) &= T(n/2) + \mcO(1) \\
         &= T(n/4) + \mcO(1) + \mcO(1) \\
         &= T(n/2^k) + \mcO(1).
  \end{align*}
  Thus our recurrence gives us a final time complexity of $ T(n) = \mcO(\log n) $.
\end{proof}

\pagebreak
\setcounter{section}{4}
\setcounter{exercise}{2}
Let $ X \tand Y $ be two sets of integers. We define the \tit{unique sums} of $ X \tand Y $ as the set of integers of the form \begin{align*}
  z = x+y
\end{align*} where $ x\in X,\ y\in Y $, and the choice of $ (x,y) \in X\times Y $ is unique. You may assume that all the integers in $ X $ are distinct (amongst themselves) and that all the integers in $ Y $ are distinct (amongst themselves).

\begin{subexercise}
  Suppose $ X\tand Y $ each have $ n $ integers. Design and analyze a $ \mcO(n\sq \log n) $ time algorithm to compute the unique sums of $ X\tand Y $.
\end{subexercise}

\begin{quote}
\defalgo{unique-sums}{X,Y}

\doc{Given two sets of integers $ X $ and $ Y $, returns the set of unique sums of $ X $ and $ Y $. Preprocessing: sort $ X $ and $ Y $.}

\begin{steps}
  \item Let $ \mcS \tand \mcU $ be empty lists
  \item For $ x \in X $: \rcomm{$ \mcO(n) $}
  \begin{steps}
    \item For $ y \in Y $: \rcomm{$ \mcO(n) $}
    \begin{steps}
      \item Let $ s = x+y $
      \item Add $ (s, (x,y)) $ to $ S $
    \end{steps}
  \end{steps}
  \item Sort $ \mcS $ by value of $ s $ \rcomm{$ \mcO(n\sq \log n) $}
  \item For integers $ i\in [1,n\sq] $:
  \begin{steps}
    \item If \Big[$ \exists\mcS[i+1] $ AND $ \mcS[i+1].s = \mcS[i].s $\Big] AND \Big[ $ \exists\mcS[i-1] $ AND $ \mcS[i-1].s = \mcS[i].s $\Big]:
    \begin{steps}
      \item Skip this iteration
    \end{steps}
    \item Add $ \mcS[i].s $ to $ \mcU $
  \end{steps}
  \item Return $ \mcU $
\end{steps}
\end{quote}

\begin{proof}[Proof of correctness.]
We first add all possible combinations of $ x\in X $ and $ y\in Y $ and add them to the list $ \mcS $. We then sort elements of $ \mcS $ by their sum attributes. Next, we iterate through $ \mcS $ and check if any adjacent elements share the same sum. If they do, we skip that iteration. Otherwise, we add the sum to the list $ \mcU $. This guarantees that only unique sums are added to $ \mcU $.
\end{proof}

\begin{proof}[Proof of complexity.]
We first iterate through all possible combinations of $ x\in X $ and $ y\in Y $, which takes $ \mcO(n\sq) $ time. We then sort $ \mcS $ by sum, which takes $ \mcO(n\sq \log n) $ time. Finally, we iterate through $ \mcS $ to find unique sums, which takes $ \mcO(n\sq) $ time. Thus, the final time complexity for \tmo{unique-sums} is \begin{align*}
  \mcO(n\sq) + \mcO(n\sq \log n) + \mcO(n\sq) = \mcO(n\sq\log n).
\end{align*}
\end{proof}
\pagebreak

\begin{subexercise}
  \item Suppose all the integers in $ X\tand Y $ are between $ 0\tand M $ for some $ M\in \N $. Design and analyze a $ \mcO(M\log M) $ time algorithm to compute the unique sums of $ X\tand Y $.
\end{subexercise}

\begin{quote}
\defalgo{unique-sums-M}{X,Y,M}

\doc{Given two sets of integers $ X $ and $ Y $ and an integer $ M $, returns the set of unique sums of $ X $ and $ Y $. Assumes use of subroutines $ F^*_n\tand F_n $ from lecture. Preprocessing: sort $ X $ and $ Y $.}

\begin{steps}
  \item Let $ A \tand B $ be arrays of length $ M+1 $, initialized to 0
  \item For $ x \in X $: \rcomm{$ \mcO(n) $}
  \begin{steps}
    \item $ A[x] = 1 $
  \end{steps}
  \item For $ y \in Y $: \rcomm{$ \mcO(n) $}
  \begin{steps}
    \item $ B[y] = 1 $
  \end{steps}
  \item Let $ \mcN $ be the smallest power of 2 greater than $ 2(M+1) $
  \item Let $ \tilde A \tand \tilde B $ be arrays of length $ \mcN $, initialized to 0
  \item For $ i \in [0,M] $: \rcomm{$ \mcO(M) $}
  \begin{steps}
    \item $ \tilde A[i] = A[i] $
    \item $ \tilde B[i] = B[i] $
  \end{steps}
  \item For $ i \in [M+1,\mcN-1] $: \rcomm{$ \mcO(M) $}
  \begin{steps}
    \item $ \tilde A[i] = 0 $
    \item $ \tilde B[i] = 0 $
  \end{steps}

  \item Let $ \hat A = F_N(\tilde A) $ and let $ \hat B = F_N(\tilde B) $ \rcomm{$ \mcO(M\log M) $}

  \item Let $ \hat C $ be an array of length $ \mcN $, initialized to 0
  \item For $ k \in [0,\mcN-1] $: \rcomm{$ \mcO(M) $}
  \begin{steps}
    \item $ \hat C[k] = \hat A[k] \cdot \hat B[k] $
  \end{steps}

  \item Let $ C_{aux} = F_N^*(\hat C) $
  \item For $ k \in [0,\mcN-1] $: \rcomm{$ \mcO(M) $}
  \begin{steps}
    \item $ C_{aux}[k] = C_{aux}[k] / \mcN $
  \end{steps}
  \item Let $ \mcU $ be an empty list
  \item For $ z\in [0,2M] $:
  \begin{steps}
    \item If $ round(C_{aux}[z]) = 1 $:
    \begin{steps}
      \item Append $ z $ to $ \mcU $
    \end{steps}
  \end{steps}
  \item Return $ \mcU $
\end{steps}
\end{quote}

\begin{proof}[Proof of correctness.]
  We form ``indicator'' arrays $A$ and $B$ of length $M+1$, then create extended copies $ \tilde A\tand \tilde B $ of length $N \ge 2(M+1)$.
Applying $F_N$ to each padded array,
we multiply their transforms pointwise and take the inverse transform $F_N^*$. We then
divide by $N$, because $F_N^*(F_N(\cdot)) = N \cdot (\cdot)$.
A well known DFT property implies that this returns the (linear) convolution of $A$ and $B$.
Hence each index $z$ of the output equals
$\sum_{x+y=z} A[x] \, B[y]$, i.e.\ the number of pairs $(x,y)$ summing to $z$.
Therefore, $z$ is a unique sum exactly if this convolution value is 1.
\end{proof}

\pagebreak

\begin{subexercise}
  \item Suppose now that we have $ k $ sets $ X_1,\ldots, X_k $, each consisting of integers between
  $ 1\tand M $. (You may again assume no duplicates within each set.) A unique sum of $ X_1,\ldots, X_k $ is defined as an integer of of the form \begin{align*}
    z = x_1 + \cdots + x_k
  \end{align*} where $ x_i\in X_i $ for all $ i\in [k] $, and the choice of $ (x_1,\ldots,x_k) \in X_1\times \cdots \times X_k $ is unique. Design and analyze an algorithm to compute the unique sums of $ X_1,\ldots,X_k $ in $ \mcO(kM\log(Mk)\log(k)) $ time. You may assume for simplicity that $ k $ is a power of 2.
\end{subexercise}


\end{document}
