\documentclass{article}

\input{preamble}
\input{letterfont}
\input{macros}

\fancyhead[L]{\tbo{Josh Park, Amy Kang, Diya Singh \\ Prof. Kent Quanrud}}
\fancyhead[C]{\tbo{CS 390ATA \\ Homework 5 (\theexercise)}}
\fancyhead[R]{\tbo{Spring 2025 \\ Page \thepage}}

\begin{document}
\setcounter{section}{11}
\setcounter{exercise}{5}

\tbo{Exercise \theexercise} Let $ \llist{x}{1}{n}\in \N $.
For each of the following problems, either (a) design and analyze a polynomial time algorithm (the faster the better), or (b) prove that a polynomial time algorithm would imply a polynomial time algorithm for SAT.\textsuperscript{\hyperref[fn:sat]{2}} \\
\noindent\rule{2in}{0.4pt} \\
\parbox{\linewidth}{\small \textsuperscript{\label{fn:sat}2}You can use the solution of one subproblem to solve another, as long as there's no circular dependencies overall.}

%-------------------------------- 11.5.1 SOLUTION ---------------------------------%

\begin{subexercise}\label{qs:ptn}
  The \tit{partition problem} asks if one can partition $ \llist{x}{1}{n} $ into two parts such that the sums of each part are equal.
\end{subexercise}

\begin{solution}
We claim that a polynomial time solution for the partition problem would imply a polynomial time solution for SAT.
To see this, we present a polynomial time reduction from subset sum, a problem known to be hard, to the partition problem.

\begin{notation}
Given some set $ S = \{\llist{s}{1}{n}\}$, we denote the sum $ \sum\limits_{s\in S}s $ with  $  \Sigma S $.
\end{notation}

Consider an arbitrary instance of subset sum.
That is, suppose we have a set of positive integers \begin{align*}
  A := \{\llist{\alpha}{1}{n}\} \sseq \N
\end{align*}
and a positive integer target value $ T\in \N $.

Now, let $ x := 2T-\Sigma A $ and define a new set $ \Abar := A\cup \{x\}$.
We claim that $A$ has a subset sum to $T$ if and only if an exact partition of $\Abar$ exists.

Note that if $ T < \Sigma A/{2} $, the value of $ x $ becomes less than 0.
However, if there exists some set $ B = \{\llist{\beta}{1}{k}\}\sseq A $ such that $ \Sigma B = T $, then $ A\setminus B $ sums to $ \Sigma A - T \leq \Sigma A/{2} $.
Thus, WLOG we can simply rephrase the problem to use $ \Sigma A - T$ as the target value instead.

Then, notice that \begin{align*}
  \Sigma \Abar &= \Sigma A + 2T-\Sigma A \\
  &= 2T.
\end{align*}
% Obviously, if there exists some valid partition of $ \Abar $ into two subsets with equal sums then each partitioned subset must sum to $ T $.
\begin{subproof}[Correctness.] (SS\imp PP)
  Suppose there exists some $ B\sseq A $ such that $ \Sigma B = T $.
  Consider the partition of $ \Abar $ defined as $ \Bbar = B\cup \{x\} $.
  Then, \begin{align*}
    \Sigma \Bbar= T + 2T-\Sigma A.
  \end{align*}
  The remaining partition is then $ C := \Abar\setminus \Bbar $, and \begin{align*}
  \Sigma C &= 2T- T + 2T-\Sigma A \\
  &= T + 2T-\Sigma A,
  \end{align*}
  and we can see that these two sums are equal.
  Hence, the partition problem is solved.

  (SS\pmi PP)
  Suppose the set $ \Abar $ has a valid partition such that each of the two subsets sum to $ T $.
  Recall that $ \Abar $ is defined as the union of $ A $ and the singleton set $ \{x\} $.
  By the pigeonhole principle, we know that one of these subsets of $ \Abar $ is a subset of $ A $, whence the subset sum problem is solved.
\end{subproof}
Since each step in the reduction process takes only $ O(1)\tor O(n) $ time, the entire reduction can be done in polynomial time relative to the size of $ A $.
Thus, a polynomial time solution for the partition problem implies a polynomial time solution for SAT.
\end{solution}
\pagebreak

%-------------------------------- 11.5.2 SOLUTION ---------------------------------%

\begin{subexercise}\label{qs:3ptn}
  The \tit{3-partition problem} asks if one can partition $ \llist{x}{1}{n} $ into 3 parts such that the sums of each part are all equal.
\end{subexercise}

\begin{solution}
  We claim that a polynomial time solution for the 3-partition problem would imply a polynomial time solution for SAT.
  To see this, we present a polynomial time reduction from the partition problem, which we showed to be hard in \ref{qs:ptn}.

  \begin{notation}
  Given sets $ A \tand B$, we denote the \tit{\href{https://en.wikipedia.org/wiki/Disjoint_union}{disjoint union}} of $ A\tand B $ by $ A\sqcup B $.
  \end{notation}

  Consider an arbitrary instance of the partition problem.
  That is, consider a set of positive integers \begin{align*}
    A := \{\llist{\alpha}{1}{m}\} \sseq \N.
  \end{align*}
  The partition problem seeks two disjoint subsets $ B,C\sseq A $ such that $ B\sqcup C = A $ and $ \Sigma B = \Sigma C $.

  Note that if $ \Sigma A $ is odd or the cardinality of $ A $ is less than 2, then the problem becomes impossible.
  Thus, WLOG we may assume that $ \Sigma A = 2n $ for some $ n\in \N $ and that $ A $ contains at least 2 elements.

  Now, let $ x := n $ and define a new set $ \Abar := A\cup \{x\}\imp \Sigma \Abar = 3n $.

  \begin{subproof}[Correctness.]
    (PP\imp 3P)
    Assume that $ \exists B\sseq A\ \suth \Sigma B = n $.
    Let $ C := A \setminus B $ and notice that $ \Sigma C = 2n - n = n $.
    By construction,
    \begin{align*}
      \Abar = C\sqcup  B \sqcup \{x\} \tand \Sigma C = \Sigma B = \Sigma \{x\}.
    \end{align*}
    Hence, the 3-partition problem is solved.

    (PP\pmi 3P)
    Assume that there exists a valid 3-partition of $ \Abar $.
    That is, assume that there exist $ A_1, A_2, A_3 \sseq \Abar $ such that \begin{align*}
      \Sigma A_1 = \Sigma A_2 = \Sigma A_3 = n \tand A_1 \sqcup A_2 \sqcup A_3 = \Abar.
    \end{align*}
    We already know $ \{x\}\sseq \Abar $ and $ x=n $, so WLOG we can set $ A_1 := \{x\} $.
    Then, we have that $ A_2 \sqcup A_3 = \Abar \setminus A_1 = A $, and we know $ \Sigma A_2 = \Sigma A_3 = n $, whence the partition problem is solved.
  \end{subproof}
  This reduction can obviously be done in polynomial time relative to the size of $ A $.
  Thus a polynomial time solution for the 3-partition problem would imply a polynomial time solution for the partition problem, which we have already shown would imply a polynomial time solution for SAT.
\end{solution}
\pagebreak

%-------------------------------- 11.5.3 SOLUTION ---------------------------------%

\begin{subexercise}
  The \tit{any-k-partition problem} asks if one can partition $ \llist{x}{1}{n} $ into $ k $ parts, for any integer $ k \geq 2 $, such that the sums of each part are all equal.
\end{subexercise}

\begin{solution}
  We claim that a polynomial time solution for the $ k $-partition problem would imply a polynomial time solution for SAT.
  To see this, we present an inductive proof of a polynomial time reduction from the 3-partition problem to the $ k $-partition problem.

  As stated, our base case will be the 3-partition problem, which we showed to be hard in \ref{qs:3ptn}.
  Assume that we have shown that the $ \ell $-partition problem is hard for all $ 3\leq\ell < k $.

  Consider an arbitrary instance of the $ (k-1) $-partition problem.
  That is, consider a set of positive integers \begin{align*}
    A := \{\llist{\alpha}{1}{m}\} \sseq \N.
  \end{align*}
  The $ (k-1) $-partition problem seeks $ k-1 $ pairwise disjoint subsets $ A_1,A_2,\ldots,A_{k-1}\sseq A $ such that \begin{align*}
    \bigsqcup_{1\leq i \leq k-1}A_{i} = A \quad \tand \quad \Sigma A_1 = \Sigma A_2 = \cdots = \Sigma A_{k-1}.
  \end{align*}
  Note that if $ \Sigma A $ is not divisible by $ k-1 $ or if the cardinality of $ A $ is less than $ k-1 $, then the problem is rendered impossible.
  Thus, WLOG we may assume that $ \Sigma A = (k-1)n $ for some $ n\in \N $ and that $ A $ contains at least $ k-1 $ elements.

  By our inductive hypothesis, we have that the existence of a polynomial time solution for the $ (k-1) $-partition problem implies the existence of a polynomial time solution for SAT.

  Now, let $ x := n $ and define a new set $ \Abar := A\cup \{x\}\imp \Sigma \Abar = kn $.

  \begin{subproof}[Correctness.]
    (($ k-1 $)P\imp $k$P)
    Assume that there exists a valid $ (k-1) $-partition for $ \Abar $.
    That is, assume that there exist $ k-1 $ pairwise disjoint subsets $ A_1,\ldots,A_{k-1}\sseq A $ such that \begin{align*}
      \bigsqcup_{1\leq i \leq k-1}A_{i} = A \quad \tand \quad \Sigma A_1 = \Sigma A_2 = \cdots = \Sigma A_{k-1}.
    \end{align*}
    By construction, we have that \begin{align*}
      \Abar = \lt[\bigsqcup_{1\leq i \leq k-1}A_{i}\rt] \sqcup \{x\} \quad \tand \quad \Sigma A_1 = \Sigma A_2 = \cdots = \Sigma A_{k-1} = \Sigma \{x\} = n.
    \end{align*}
    Hence, the $ k $-partition problem is solved.

    (($ k-1 $)P\pmi $k$P)
    Assume that there exists a valid $ k $-partition of $ \Abar $.
    That is, assume there exist $ k $ pairwise disjoint subsets $ A_1, A_2, \ldots, A_k \sseq \Abar $ such that \begin{align*}
      \Abar = \bigsqcup_{1\leq i \leq k}A_{i} \quad \tand \quad \Sigma A_1 = \Sigma A_2 = \cdots = \Sigma A_{k} = n.
    \end{align*}
    We already know $ \{x\}\sseq \Abar $ and $ x=n $, so WLOG we can set $ A_1 := \{x\} $.
    Then, we have that \begin{align*}
      \bigsqcup_{2\leq i \leq k}A_{i} = \Abar \setminus A_1 = A \quad \tand \quad \Sigma A_2 = \Sigma A_3 = \cdots = \Sigma A_{k} = n,
    \end{align*}
    whence the $ (k-1) $-partition problem is solved.
  \end{subproof}
  This reduction can obviously be done in polynomial time relative to the size of $ A $.
  Thus, a polynomial time solution for the $k$-partition problem implies a polynomial time solution for the $ (k-1) $-partition problem, and by induction does so for the 3-partition problem (and equivalently for SAT).
\end{solution}
\pagebreak

%-------------------------------- 11.5.4 SOLUTION ---------------------------------%

\begin{subexercise}
  The \tit{almost-partition problem} asks if one can partition $ \llist{x}{1}{n} $ into two parts such that the two sums of each part differ by at most 1.
\end{subexercise}

\begin{solution}
We claim that a polynomial time solution for the almost-partition problem would imply a polynomial time solution for SAT. To see this, we present a polynomial time reduction from the partition problem, which we proved in \ref{qs:ptn} to be hard.

Suppose we want to solve the partition problem on a set of positive integers $A := \{ a_1, ..., a_n\} \subseteq \mathbb{N}$, given a solution to the almost-partition problem as a blackbox. We transform $A$ into the the set
\[ A' := \{ 2a_i : a_i \in A\} = \{ 2a_1, ..., 2a_n\} \]
and then apply the almost-partition solution to $A'$. Since all elements in $A'$ are even, it is impossible for partitions to differ by exactly 1. Hence, we claim  $A$ has a partition if and only if $A'$ has an almost-partition.


\vspace{1.5cc}
\begin{subproof} [Correctness]
To prove correctness, let us first assume that $A'$ has an almost-partition; that is, there exists some $B' \subseteq A'$ for which
\[\sum B' = \sum \ (A' \smallsetminus B') \hspace{1cc} \text{or} \hspace{1cc} \sum B' = \sum \ (A' \smallsetminus B') \pm 1\]
 $B'$ and $A' \smallsetminus B'$ are both subsets of $A'$, so we have $2 \divs \sum B'$ and $2 \divs \sum (A' \smallsetminus B')$.

 Since $\sum B' = \sum \ (A' \smallsetminus B') \pm 1$ cannot be true, we must have $\sum B' = \sum \ (A' \smallsetminus B')$, which can be rewritten $2 \sum B' = \sum A'$.

Let $B := \{a_i : 2a_i \in B'\} \subseteq A$. Then $2 \sum B = \sum B' = \frac{1}{2} \sum A' = \sum A$; hence, $A$ has an exact partition.

\vspace{1cc}
Conversely, we now assume that $A$ has an exact partition given by $\sum B = \sum (A \smallsetminus B)$ for some $B \subseteq A$. If we define $B' := \{2a_i : a_i \in B\} \subseteq A$, then $A'$ also has an exact partition given by $\sum B' = \sum (A' \smallsetminus B')$ which is, by definition, an almost-partition of $A'$.
\end{subproof}


This reduction can clearly be performed in polynomial time relative to the input size of $A$ and expression size of the integers in $A$. Since we proved above that almost-partition can be used to solve the exact partition problem, which is known to be hard, we can conclude that a polynomial time solution for the almost-partition problem would also imply a polynomial-time solution for SAT.
\end{solution}
\pagebreak

%-------------------------------- 11.5.5 SOLUTION ---------------------------------%

\begin{subexercise}
  \footnote[3]{IMO, this one is the trickiest.}Let $n$ be even.
  The \tit{perfect partition problem} asks if one can partition $ \llist{x}{1}{n} $ into two parts such that
  \begin{enumerate}[label=(\alph*)]
    \item Each part has the same sum.
    \item Each part contains the same number of $ x_i $'s.
  \end{enumerate}
\end{subexercise}
\begin{solution}
  We claim that a polynomial time solution for the partition problem would imply a polynomial time solution for SAT.
  To see this, we present a polynomial time reduction from the partition problem, a problem known we showed to be hard in \ref{qs:ptn}, to the perfect partition problem.

  Consider an arbitrary instance of the partition problem.
  That is, consider a set of positive integers \begin{align*}
    A := \{\llist{\alpha}{1}{m}\} \sseq \N.
  \end{align*}
  The partition problem seeks two disjoint subsets $ B,C\sseq A $ such that $ B\sqcup C = A $ and $ \Sigma B = \Sigma C = \Sigma A / 2$.

  Note that if $ \Sigma A $ is odd or the cardinality of $ A $ is less than 2, then the problem becomes impossible.
  Thus, WLOG we may assume that $ \Sigma A = 2T $ for some $ T\in \N $ and that $ A $ contains at least 2 elements.

  In any valid partition of A, the two parts may have different cardinalities.
  Suppose that in a given partition we have $ |B| = p $ and $ |C| = q $.
  WLOG we can assume $ p \geq q $, and let the difference between the cardinalities be $ x = p-q $.

  Note that if $ x = 0 $, then the partition is already `perfect'.
  However if $ x > 0 $, although A is partitionable, it is not perfectly partitionable.
  We can't yet know the value of $ x $, but it is easy to see that it is bounded above by $ m $.
  Then for each possible $ x\in \{1,\ldots,m\}$, define $ A_x = A \cup S_x \cup Y_x $ where $ S_x := \{\llist{s}{1}{x+1} \} $ such that $ s = 1 $ for each $ s\in S_x $ and $ Y_x := \{x+1\} $.

  We claim that a partition on $A$ exists if and only a perfect partition on $A_x$ exists for some value of $x$, where $ x\in \{1,\ldots,m\}$.

  \begin{subproof}[Correctness.]
    (PP\imp PPP)
    Assume that $ \exists B,C\sseq A\ \suth B\sqcup C = A \tand \Sigma B = \Sigma C = T $. Let $ x=\order{B}-\order{C} $

    \begin{subproof}[Case 1 ($ x = 0 $).]
      As above, this case is trivial. The partition is already perfect and hence the perfect partition problem is solved.
    \end{subproof}
    \begin{subproof}[Case 2 ($ x \geq 1 $).]
      Construct $ A_x $ as above.
      Now, consider the partition of $ A_x $ in which $ B_x := B\cup Y $ and $ C_x := C\cup S_x $.
      Then $ \Sigma B_x = T + (x+1) $ and $ \Sigma C_x = T + (x+1) $.
      Also, $ \order{B_x} = \order{B}+1 $ and $ \order{C_x} = \order{C}+(x+1) $.
      However notice that by definition of $ x $, $ \order{C_x} = \order{C} + (\order{B} - \order{C} + 1) = \order{B} + 1 = \order{B_x} $ and thus the sets have equal cardinality.
      Hence the perfect partition problem is solved.
    \end{subproof}

    (PP\pmi PPP)
    Assume that there exist disjoint $ B,C\sseq A $ such that
    \begin{align*}
      B\sqcup C = A, \qquad \Sigma B = \Sigma C = n, \quad \tand \quad \order{B}=\order{C}.
    \end{align*}
    Clearly, the partition problem is solved.
  \end{subproof}
  Each step of the reduction process can be done in polynomial time relative to the size of $ A $, so it follows that the a polynomial time solution to the perfect partition problem implies a polynomial time solution to the partition problem, whence a polynomial time solution to SAT.
\end{solution}
\pagebreak

\end{document}
