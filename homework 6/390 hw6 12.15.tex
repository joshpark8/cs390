\documentclass{article}

\input{preamble}
\input{letterfont}
\input{macros}

\fancyhead[L]{\bd{Josh Park, Amy Kang, Diya Singh \\ Prof. Kent Quanrud}}
\fancyhead[C]{\bd{CS 390ATA \\ Homework 6 (\theexercise)}}
\fancyhead[R]{\bd{Spring 2025 \\ Page \thepage}}

\begin{document}
\setcounter{section}{12}
\setcounter{exercise}{15}
\bd{Exercise \theexercise.} \bd{It's-a me, Mario!}
This problem is inspired by Super Mario World, where Mario must make it from some starting point to the flag in front of a castle, collecting as many coins as possible along the way.

Let \( G = (V,E) \) be a directed graph where each vertex \( v \in V \) has \( C_v \geq 0 \) coins.
For a walk in \( G \), we say that the \it{number of coins collected by the walk} is the total sum of \( C_v \) over all \it{distinct} vertices \( v \) in the walk.
(If we visit a vertex \( v \) more than once, we still only get \( C_v \) coins total.)
Given \( s, t \in V \), the goal is to compute the maximum number of coins collected by any \( (s, t) \)-walk.

% >>==========|| 12.15 ||==========<<


% --------------------------------- 12.15.1 -------------------------------- %

\begin{subexercise} \label{mario-dag}
Let G be a DAG.
For this problem, either (a) design and analyze a polynomial time algorithm (the faster the better), or (b) prove that a polynomial time algorithm would imply a polynomial time algorithm for SAT.
\end{subexercise}

\begin{solution}
We claim that the following algorithm suffices.

\begin{quote}%
\defalgo{mcw-dag}{v, t}

\doc{given \( v, t \in V \) in a DAG, computes the maximum number of coins collected by any (v,t)-walk and returns \( -\infty \) if no such walk exists.}

\begin{steps}
    \item If \( v = t \) then return \( C_v \).
    \item Let \( m \leftarrow -\infty \).
    \item For \( (v,w)\in \delta^+(v) \) do
    \begin{steps}
        \item \( m \leftarrow \max \ (m, \ C_v + \subcall{mcw-dag}{w, t}) \)
    \end{steps}
  \item Return \( m \)
\end{steps}
\end{quote}

We find the maximum number of coins collected along any \( (s,t) \)-walk by calling \subcall{mcw-dag}{s, t}.


\begin{subproof} [Runtime]
Given a goal vertex \( t \in V \), if we cache the return value of \subcall{mcw-dag}{v, t} for all \( v \in V \), then our algorithm has the following runtime complexity:
\begin{align*}
    O \left( \sum_{v\in V} ( 1 +  d^+(v) ) \right) = O \left( m+n \right)
\end{align*}
\vspace{-0.5cc}
\end{subproof}

\vspace{-0.5cc}
\begin{subproof} [Correctness]
The correctness of this algorithm follows from performing induction, in reverse topological order, according to the recursive specification. We claim that \subcall{mcw-dag}{v,t} returns the maximum number of coins collected by any \( (v,t) \)-walk for all \( v \in V \).

In the base case, \( v \) is the last vertex in topological order---a sink---so the maximum number of collectible coins is \( C_v \) if \( v=t \) and \( -\infty \) if \( v \not= t \), as returned in the algorithm.

Now we assume that for some \( v \in V \), the claim holds for all vertices that follow \( v \) in topological order. In the case where \( v \) is a sink, there are no vertices reachable from \( v \) and consequently, the algorithm correctly returns \( C_v \) if \( v=t \) and \( -\infty \) if \( v \not= t \). Otherwise, since the claim holds, by assumption, for all \( w \) such that \( (v',w) \ \in \ \delta^{+}(v') \), we can conclude that the maximum number of collectible coins on a \( (v, t) \)-walk is
\begin{align*}
    C_{v} \ + \max_{w : \ (v,w) \ \in \ \delta^{+}(v)} \subcall{mcw-dag}{w, t}
\end{align*}
as computed in the algorithm.

Thus, by induction, the claim that \subcall{mcw-dag}{v, t} returns the maximum collectible coins along a \( (v,t) \)-walk holds for all \( v \in V \).
\end{subproof}

\end{solution}
\pagebreak


% --------------------------------- 12.15.2 -------------------------------- %

\begin{subexercise}
Let G be a general directed graph.
For this problem, either (a) design and analyze a polynomial time algorithm (the faster the better), or (b) prove that a polynomial time algorithm would imply a polynomial time algorithm for SAT.
\end{subexercise}

\begin{solution}
When \( G \) is a general directed graph, we compress \( G \) to its condensation graph \( G' \), then apply the algorithm given in \ref{mario-dag} to \( G' \).

\begin{quote}%
\defalgo{mcw}{s, t}

\doc{given \( s, t \in V \) in a general digraph, computes the maximum number of coins collected by any (s,t)-walk and returns \( -\infty \) if no such walk exists.}

\doc{see \ref{mario-dag} for \subcall{mcw-dag}{} pseudocode.}

\begin{steps}
    \item Let \( \{S_1, S_2, ..., S_k\} \) be the strongly-connected components in \( G \).
    \item Contract each \( S_i \) into a single vertex \( s_i \) in \( G' = (V', E') \) with \( C_{S_i} = \sum_{ v\in S_i} C_v \).
    \item For \( i \not= j \), \( (s_i, s_j) \in E' \) iff there exist \( u \in S_i \), \( v \in S_j \) with \( (u,v) \in E \).
    \item Let \( S_s \) and \( S_t \) be the sccs containing \( s \) and \( t \), respectively.
    \item Return \subcall{mcw-dag}{s_s, s_t}.
\end{steps}
\end{quote}

Calling \subcall{mcw}{s, t} gives us the maximum number of coins collected by any (s,t)-walk.


\begin{subproof}[Runtime]
The runtime of SCC-compression on \( G \) is in \( O(m+n) \), as is the runtime of \subcall{mcw-dag}{}. Hence, the total runtime complexity of our algorithm is in \( O(m+n) \).
\end{subproof}


\begin{subproof}[Correctness]
Since the correctness of \subcall{mcw-dag}{} was proven in \ref{mario-dag}, it suffices for us to show that our SCC-compression preserves the maximal number of coins collected along any \( (s,t) \)-walk.

First, we claim that a maximal coin-collection walk in \( G \) necessarily collects all of the coins in every SCC it visits; otherwise, we would be able to make ``detours'' in at least one SCC to pick up coins we've missed, contradicting the maximality of the walk. We call such a walk an SCC-walk.

Since we now know the maximum number of coins collected from an \( (s,t) \)-walk in \( G \) can be given by an SCC-walk on \( G \); so, we can simply maximize coin collection on only the SCC-walks in \( G \), rather than on all walks.

We have a clear correspondence between coins collected on SCC-walks in \( G \) and all walks in \( G' \):
\begin{enumerate}
    \item [(i)] First, for every \( (s,t) \)-SCC-walk \( W \) on \( G \), there exists a walk of the same coin-numerage from \( s_s \) to \( s_t \) in \( G' \) by taking all the SCCs \( W \) passes through.
    \item [(ii)] Conversely, for every walk from \( s_s \) to \( s_t \) in \( G' \), there exists an \( (s,t) \)-SCC-walk in \( G \) with the same total weight.
\end{enumerate}

By this bijection, we can conclude that the maximum coin collection over all \( (s,t) \)-walks in \( G \) is equal to the maximum coin collection over all \( (s_s,s_t) \)-walks in \( G' \).
\end{subproof}

\end{solution}
\pagebreak
\end{document}
