\documentclass{article}

\input{preamble}
\input{letterfont}
\input{macros}

\fancyhead[L]{\tbo{Josh Park, Amy Kang, Diya Singh \\ Prof. Kent Quanrud}}
\fancyhead[C]{\tbo{CS 390ATA \\ Homework 6 (\theexercise)}}
\fancyhead[R]{\tbo{Spring 2025 \\ Page \thepage}}

\begin{document}
\setcounter{section}{13}
\setcounter{exercise}{3}
% >>==========|| 13.4 ||==========<<
\begin{exercise}
Let $ G=(V,E) $ be a directed graph.
We say that a set of vertices is \tit{almost independent} if each $ v\in S $ has at most one neighbor in $ S $.\footnote[5]{Two vertices $u \tand v$ are neighbors if they are connected by an edge.}
Consider the problem of computing the maximum carcdinality of any almost independent set of vertices.
For this problem, either (a) design and analyze a polynomial time algorithm (the faster the better), or (b) prove that a polynomial time algorithm would imply a polynomial time algorithm for SAT.
\end{exercise}

\begin{solution}
  We claim that the maximum cardinality almost independent set (MCAIS) problem is NP-complete.
  To see this, we present a polynomial time reduction from the maximum cardinality independent set (MCIS) problem, a problem we saw to be NP-complete in class, to MCAIS.

  \begin{note}
    Since that the condition for vertices $ v,w\in V $ to be neighbors only requires that there be a connection between the two, we need not be concerned with the directional property of the edges in $ E $.
    Hence, we will interpret $ G $ as an undirected graph.
  \end{note}

  Consider an arbitrary instance of independent set.
  That is, suppose we have a graph $ G=(V,E) $.
  The MCAIS problem seeks the largest $ S\ss V $ such that $ (v,w)\not\in E $ for all $ v,w\in S $.

  Create an auxiliary graph $ G'=(V',E') $, where $ V'=\bigcup\limits_{v_i\in V}\{v_i,v'_i\} $ and $ E'=\{(v_i,v_i')\mid v_i\in V\}\cup E $.
  Thus, $ V' $ contains duplicates of each vertex in $ V $ and $ E' $ contains each edge in $ E $, as well as an additional edge connecting each pair of vertices $ v_j $ and $ v_j' $.
  We can think of $ G' $ as a bilayered graph, where $ G $ is the top layer and every vertex has a copy of itself `hanging' below.
  Importantly, a `hanging' vertex is only connected to the original vertex.

  \begin{subproof}[Correctness.]
    (MCIS\imp MCAIS)
    Assume we have that a maximum cardinality independent set of $ G $ has cardinality $ n $.
    Notice that a subset $ S' $ of $ V' $ containing $ v'_i $ for every $ i $ is independent over $ G' $ with cardinality $ \order V $, since no two hanging vertices are connected by an edge.
    We know there exists some independent set $ S $ of cardinality $ n $ over $ V $.
    Then $ S'' = S\cup S' $ must be almost independent over $ G' $ with cardinality $ \order V + n $, since each vertex in $ S' $ gains at most one neighbor in the top layer of $ G' $.
    Now assume ad absurdum that we have some $ H\sseq V' $ such that $ \order{H} > \order{S''} $.
    If there exists some hanging vertex $ v'_k\not\in H $, then it must be the case that $ v_k\in H $ with some neighboring vertex $ v_{k+1}\in H $.
    However, this implies that $ v'_{k+1}\not\in H $, so we can always replace $ v_k,v_{k+1} $ with their hanging counterparts.
    Thus WLOG every hanging $ v'_i\in H $.
    So $ \order H = \order V + m $ and by hypothesis $ \order V + m > \order V + n $.
    Then we can remove the hanging vertices from $ H $ to get an independent set over $ G $ with cardinality $ m > n $.
    However, this contradicts that the maximum cardinality of an independent set over $ G $ is $ n $.
    Thus $ S'' $ must be a maximum cardinality almost independent set of $ G' $ with cardinality $ \order V + n $.

    (MCIS \pmi MCAIS)
    Assume we have that a maximum cardinality almost independent set $ S $ of $ G' $ has cardinality $ n $.
    As above, we can say WLOG $ S $ contains every hanging vertex.
    Assume ad absurdum that $ S $ contains adjacent vertices $ v_i,v_{i+1} \in V $.
    But $ v'_i \in S $ whence $ v_i $ has two neighbors, a contradiction.
    Thus no two vertices in $ S $ are adjacent in the top layer of $ G' $, which we know to be isomorphic to $ G $.
    Hence by removing every hanging vertex from $ S $, we can get an independent set $ H $ over $ G $ of cardinality $ \order H = n - \order V $.
    Once again, assume ad absurdum that there exists some independent set $ K $ over $ G $ with cardinality $ m > n-\order V $.
    Then by adding every hanging vertex to the set, we obtain an almost independent set with cardinality $ m+\order V > \order H + \order V = n $.
    However, this contradicts that $ n $ is the maximum cardinality for an almost indepedent set over $ G' $ whence $ H $ must be a maximum cardinality independent set over $ G $.
  \end{subproof}
\end{solution}
\pagebreak

\end{document}
