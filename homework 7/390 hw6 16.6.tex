\documentclass{article}

\input{preamble}
\input{letterfont}
\input{macros}

\fancyhead[L]{\tbo{Josh Park, Amy Kang, Diya Singh \\ Prof. Kent Quanrud}}
\fancyhead[C]{\tbo{CS 390ATA \\ Homework 7 (\theexercise)}}
\fancyhead[R]{\tbo{Spring 2025 \\ Page \thepage}}

\begin{document}
\tbo{Exercise 16.6} Let \( G=(V,E) \) be an undirected graph with distinct nonnegative edge weights \( w:E\to\R \).
For a spanning tree \( T \), we say that the \tit{bottleneck weight of \( T \)} is the maximum weight edge in \( T \), \( \max_{e\in T} w(e) \).
% >>==========|| 16.6 ||==========<<
\setcounter{section}{16}
\setcounter{exercise}{6}
\begin{subexercise}
Prove that the MST is also a minimum bottleneck weight spanning tree of \( G \).
\end{subexercise}

\begin{solution}

\end{solution}
\pagebreak

\begin{subexercise}
  Design and analyze a \( O(m + n) \)-time algorithm for computing a minimum
bottleneck weight spanning tree of \( G \).
(This is faster than any of our algorithms for MST.)\footnote[4]{Here's step 1: compute the median edge weight in \( O(m) \) time.}
\end{subexercise}

\begin{solution}

\end{solution}
\pagebreak

\end{document}
