\documentclass{article}

\input{preamble}
\input{letterfont}
\input{macros}

\fancyhead[L]{\bd{Josh Park, Amy Kang, Diya Singh \\ Prof. Kent Quanrud}}
\fancyhead[C]{\bd{CS 390ATA \\ Homework 7 (\theexercise)}}
\fancyhead[R]{\bd{Spring 2025 \\ Page \thepage}}

\begin{document}
\setcounter{section}{17}
\setcounter{exercise}{2}
% >>==========|| 17.3 ||==========<<
\begin{exercise}
Let \( G=(V,E) \) be an undirected graph, and let \( a, b, c \in V \) be three distinct vertices.
We define an \( (a, b, c) \)-path as a path from \( a \) to \( c \) that goes through \( b \).
Consider the problem of deciding if there exists an \( (a, b, c) \)-path.
For this problem, either (a) design and analyze a polynomial time algorithm (the faster the better), or (b) prove that the problem is NP-Hard.
\end{exercise}

\begin{solution}

This problem has a polynomial-time solution.

Abstractly, if we can find two simple paths in $G$---one from $a$ to $b$ and one from $b$ to $c$---that are vertex-disjoint except for at vertex $b$, then we can concatenate these two paths to get an $(a,b,c)$-path.

To do this, we construct an auxiliary digraph $G'$.

First, to force vertex-disjointedness, we split each vertex $v \in V$ into two vertices, $v_{in}$ and $v_{out}$ and assign a directed edge from $v_{in}$ to $v_{out}$. Then for each undirected edge $\{u, v\}$ in $G$, we add two directed edges $(u_{out}, v_{in})$ and $(v_{out}, u_{in})$.
Finally, we assign $b_{out}$ to be the source vertex and direct both $a_{out}$ and $c_{out}$ to a single destination vertex $t$.

Now, finding two edge-disjoint paths from $b_{out}$ to $t$ in $G'$ will guarantee vertex-disjoint paths from $b$ to $a$ and $b$ to $c$ in $G$. The former path can be reversed and concatenated to obtain an $(a, b, c)$-path.

\begin{quote}%
\defalgo{abc-path}{G, a, b, c}

\doc{given $a, b, c \in G$ in an undirected graph, computes whether an $(a,b,c)$-path exists in $G$.}

\begin{steps}
    \item Construct the auxiliary digraph $G'$:
    \begin{steps}
        \item For every vertex $v \in V$, create vertices $v_{in}$ and $v_{out}$ and add the edge ($v_{in}, v_{out}$).
        \item For every undirected edge $\{u,v\} \in E$, add the edges ($u_{out}, v_{in}$) and ($v_{out}, u_{in}$).
        \item Let $s \leftarrow b$.
        \item Create a sink vertex $t$, and add edges from $a_{out}$ and $c_{out}$ to $t$.
    \end{steps}
    \item If \subcall{augmenting-paths}{G', s, t} = 2 then
    \begin{steps}
        \item Return \texttt{true}.
    \end{steps}
  \item Return \texttt{false}.
\end{steps}
\end{quote}


\vspace{1cc}
\begin{subproof} [Runtime.]
We use \subcall{augmenting-paths}{G', s, t} to compute the maximal number of edge-disjoint $(s,t)$-paths. Note that we can slightly modify \subcall{augmenting-paths}{} to return true immediately when we find 2 such paths so that the runtime is in $O(m)$.
\end{subproof}


\begin{subproof} [Correctness.]
We now prove that an $(a,b,c)$-path exists in $G$ if and only if \subcall{augmenting-paths}{G', s, t} = 2.

First, suppose there are two edge-disjoint paths from $b_{out}$ to $t$ in $G'$.
Since the only incoming vertices to $t$ are $a_out$ and $c_out$, one path translates to a path that goes to vertex $a \ (a_{in}, a_{out})$ and other, a path that goes to vertex $c \ (c_{in}, c_{out})$ from $b$. Vertex-splitting guarantees that these two paths are vertex-disjoint in $G$; hence, there exist vertex-disjoint paths from $b$ to $a$ and $b$ to $c$. The path from $b$ to $a$ can be reversed, since $G$ is undirected, and prepended to the path from $b$ to $c$ to get an $(a, b, c)$-path.

Conversely, suppose $G$ has an $(a,b,c)$-path.
This path can be split in to two vertex-and-edge-disjoint paths; one from $b$ to $a$ and the other, from $b$ to $c$. It follows that these two paths can be extended to two edge-disjointed paths from $b_{out}$ to $t$ in $G'$. Since the maximum path-packing in $G'$ clearly has size 2, \subcall{augmenting-paths}{G', b_{out}, t} will return 2.
\end{subproof}


\end{solution}
\pagebreak

\end{document}
