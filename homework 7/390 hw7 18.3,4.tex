\documentclass{article}

\input{preamble}
\input{letterfont}
\input{macros}

\fancyhead[L]{\bd{Josh Park, Amy Kang, Diya Singh \\ Prof. Kent Quanrud}}
\fancyhead[C]{\bd{CS 390ATA \\ Homework 7 (\theexercise)}}
\fancyhead[R]{\bd{Spring 2025 \\ Page \thepage}}

\begin{document}
% >>==========|| 18.3 ||==========<<
\setcounter{section}{18}
\setcounter{exercise}{2}
\begin{exercise} \label{ex:flow-decomp}
  Suppose you had an \( (s, t) \)-flow \( f \).
  We know that there exists an \( (s,t) \)-path packing of the same size as \( f \); here we are interested in algorithms that take \( f \) and compute such a path packing.
  Such a path packing is called a \it{flow decomposition} of \( f \).

  Design and analyze an algorithm that, in \( O(m\sq) \) time, computes a maximum path packing \( x \) of the same size as \( f \), such that:
  \begin{enumerate}
    \item There are at most \( m \) distinct paths (with nonzero value) in \( x \).
    \item If \( f \) is integral, then \( x \) is also integral.
  \end{enumerate}
\end{exercise}

\begin{solution}

\end{solution}
\pagebreak

% >>==========|| 18.4 ||==========<<
\bd{Exercise 18.4.} This exercise develops a \( O((m\sq + mn\log(n))\log(\lambda)) \)-time algorithm for maximum \( (s,t) \)-flow and builds on ideas from exercise \ref{ex:flow-decomp}.

\stepcounter{exercise}
\begin{subexercise}
  Prove the following: Given any \( (s, t) \)-flow problem with max flow value \( \lambda > 0 \), there exists an \( (s, t) \)-path where the minimum capacity edge is at least \( \lambda/m \).
\end{subexercise}

\begin{solution}

\end{solution}
\pagebreak

\begin{subexercise}
  Describe an \( O(m+n\log(n)) \)-time algorithm to find the path described above.\footnote[4]{\( O(m\log n) \) time is a little easier and this running time would still get partial credit.
    Even if the \( O(m + n log(n)) \)-running time eludes you, you can assume it as a black box for the next part.}
\end{subexercise}

\begin{solution}

\end{solution}
\pagebreak

\begin{subexercise}
  Prove the following: Given any \( (s, t) \)-flow problem with max flow value \( \lambda > 0 \), there exists an \( (s, t) \)-path where the minimum capacity edge is at least \( \lambda/m \).
\end{subexercise}

\begin{solution}

\end{solution}
\pagebreak

\end{document}
