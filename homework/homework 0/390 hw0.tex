\documentclass[10pt]{article}
\usepackage[utf8]{inputenc}

% \usepackage[margin=1in]{geometry} %One inch margins

%Makes things a little prettier
\usepackage{lmodern}%
\usepackage[T1]{fontenc}%
% \usepackage{microtype}%
\usepackage[varqu,varl]{inconsolata} % sans serif typewriter

% Bring in the math!
\usepackage{amsmath}%
\usepackage{amsthm}%
\usepackage{amssymb}%
\usepackage{amsfonts}%
\usepackage{inconsolata}%
% mleftright is helpful for making the parenthesis (and other delimiters) fit their contents height-wise.
\usepackage{mleftright}%
% The following macro puts parenthesis around the contents using mleftright and adapt the height.
\newcommand{\parof}[1]{\mleft( #1 \mright)}
% Of course you can define similar commands for other delimiters.

% To typeset algorithms I simply use numbered lists (e.g., enumerate). I use nested lists for inner clauses. The following is a prettier version of this approach that I use in my notes; it changes the font of the numbering to typewriter font (as a visual cue), normalizes some margins, and so forth. I often put the whole algorithm inside a \begin{quote}...\end{quote} to help set it apart from the text. You can use labels and references (as usual) to refer to steps.

\usepackage{enumitem}%

\input{preamble}
\input{letterfont}
\input{macros}

\fancyhead[L]{\tbo{Josh Park \\ Prof. Quanrud}}
\fancyhead[C]{\tbo{CS 390 ATA \\ Advanced Topics in Algorithms}}
\fancyhead[R]{\tbo{Spring 2025 \\ Homework 0 (corrections)}}

\begin{document}
% \begin{note}
% For brevity's sake, I have assumed the Python convention for default parameter values  in the solutions for Problems 1.3 and 2.2. That is, assuming we have some function defined as

% \tul{def func \tmo{someExampleFunction$(arg1, arg2 := 1, arg3 := 2)$:}},

% we assume that

% \tmo{someExampleFunction}(\( \alpha \), \( \beta \), \( \gamma \)) \qimp
%   \( arg1 = \alpha \), \ \( arg2 = \beta \), \ \( arg3 = \gamma \),\\
% \tmo{someExampleFunction}(\( \alpha \), \( \beta \)) \hspace{1.177em}\qimp
%   \( arg1 = \alpha \), \ \( arg2 = \beta \), \ \( arg3 = 2 \),\\
% \tmo{someExampleFunction}(\( \alpha \)) \hspace{2.077em} \qimp
%   \( arg1 = \alpha \), \ \( arg2 = 1 \), \ \,\( arg3 = 2 \).

% If this is not acceptable, please let me know.
% \end{note}

\tbo{Exercise 1.3.1.} % PROBLEM 1.3.1
Let $A[1..n]$ be in a \tit{rotated} sorted order. That is, there exists an index $i \in [n]$, called the \tit{rotation index}, such that the concatenation of $A[i..n]$ and $A[1..i - 1]$ is sorted in increasing order. (One can think of $A[1..n]$ as being sorted initially, and then rotated cyclically to the right by $i - 1$ slots). The goal is to compute the unknown rotation index $i$. For simplicity, you may assume all the elements are distinct.

\begin{solution}\ % SOLUTION 1.3.1
\begin{quote}%
\noindent\tul{def func \tmo{computeRotationIndex($A[1..n]$):}}%

\doc{Given an array of length \( n \) in `rotated' sorted order, return the `rotation'
index.
}

\begin{steps}
  % Base case
  \item If $ n \leq 1 $, return $ 1 $.
  % Split the search field
  \item Let $ k := \floor{\frac{n}{2}} $.
  \item If $ \parof{A[mid] > A[n]} $:
  \begin{steps}
    \item Return $ k$ + \tmo{computeRotationIndex}$(A[k+1..n])$.
  \end{steps}
  \item Else if $ \parof{k = 1} $ OR $ \parof{A[k] > A[n] }$:
  \begin{steps}
    \item Return $ k $
  \end{steps}
  \item Else:
  \begin{steps}
    % Pivot lies in [left..mid]
    \item Return \tmo{computeRotationIndex}$(A[1..k-1])$
  \end{steps}
\end{steps}
\end{quote}
\end{solution}
\pagebreak

\tbo{Exercise 1.3.2.} % PROBLEM 1.3.2
We say that $A[1..n]$ is a \tit{mountain} if there exists an index $i \in [n]$, called the
\tit{peak}, such that $A[1..i]$ is sorted in increasing order and $A[i + 1..n]$ is sorted in
decreasing order. Given a mountain $A[1..n]$, find the peak. For simplicity, you may assume all
elements are distinct.

\begin{solution}\ % SOLUTION 1.3.2
\begin{quote}%
\noindent\tul{def func \tmo{findPeak($A[1..n]$):}}%

\doc{Given a `mountain' of length \( n \), find the index of the `peak'.}

\begin{steps}
\item If \( \parof{n \leq 1} \), return 1
  \item Let \( k := \floor{\frac{n}{2}} \)
  \item If \( \parof{A[k] < A[k + 1]} \): %\rcomm{ascending; peak is to the right}
    \begin{steps}
    \item Return $ k $ + \tmo{findPeak}(\( A[k+1..n]\))
    \end{steps}
  \item Else if $\parof{k > 1}$ AND $\parof{ A[k-1] < A[k] }$:
    \begin{steps}
    \item Return $ k $
    \end{steps}
  \item Else:
    \begin{steps}
    \item Return \tmo{findPeak}(\( A[1..k] \)):
    \end{steps}
\end{steps}
\end{quote}
\end{solution}
\pagebreak

\tbo{Exercise 1.3.3.} % PROBLEM 1.3.3
We say an index $i \in [n]$ is a \tit{local minimum} if either
\begin{enumerate}[label=(\alph*)]
  \item $i = n = 1$,
  \item $i = 1$ and $A[1] \leq A[2]$,
  \item $i = n$ and $A[n] \leq A[n-1]$, or
  \item $1 < i < n$, $A[i-1] \geq A[i]$ and $A[i] \geq A[i + 1]$.
\end{enumerate}
The goal is to find a local minimum from an array $A[1..n]$.

\begin{solution}\ % SOLUTION 1.3.3
\begin{quote}%
\noindent\tul{def func \tmo{findLocalMinimum($A[1..n]$):}}%

\doc{Given an array of length \( n \), identify the `local minimum'.}%

\begin{steps}
% Base case
\item If $ (n = 1) $: return 1 \rcomm{(a)}
\item Let $ k := \floor{\frac{n}{2}} $
% Check if k is a local minimum
\item If \big[$(k = 1)$ AND $(A[k] \leq A[k + 1])$\big] OR \rcomm{(b)}\\
  \nf{\quad}\big[$(k = n)$ AND $\big(A[k] \leq A[k - 1]\big)$\big] OR \rcomm{(c)}\\
  \nf{\quad}\big[$(1 < k < n)$ AND $(A[k - 1] \geq A[k])$ AND $(A[k] \leq A[k + 1])$\big]: \rcomm{(d)}
  \begin{steps}
  \item Return $ k $
  \end{steps}
\item If \big[$ (k > 1) $  AND $ A[k-1] < A[k] $\big]:
  \begin{steps}
  \item Return \tmo{findLocalMinimum}$(A[1..k-1])$
  \end{steps}
\item Else:
  \begin{steps}
  \item Return \tmo{findLocalMinimum}$(A[k+1..n])$
  \end{steps}
\end{steps}
\end{quote}
\end{solution}
\pagebreak

\tbo{Exercise 1.3.4.} % PROBLEM 1.3.4
Prove that each of the problems above have a lower bound of \( \Omega(\log n) \) queries in the comparison model.

\begin{solution}\ % SOLUTION 1.3.4
Each of the problems above have $ n $ possible values for the specified index. Any given comparison can have 2 possible outcomes: \boolT~or \boolF~. Thus in the worst case scenario we have \( 2^k \geq n \) where \( k \) is the number of queries. Solving for \( k \) gives us \( k \geq \log_2 n \). Therefore, the lower bound is \( \Omega(\log n) \).
\end{solution}
\pagebreak

\tbo{Exercise 2.1.2.} % PROBLEM 2.1.2
In the \tit{cyclic towers of Hanoi} problem, a ring can only be moved in ``cyclic'' order from post $A$ to post $B$, from post $B$ to post $C$, and from post $C$ to post $A$. The goal is to move $n$ sorted rings from post $A$ to post $B$.

\begin{solution}\ % SOLUTION 2.1.2
\begin{quote}
\noindent\tul{def func \tmo{cyclic-Hanoi($n$, $from$, $to$):}}%

\doc{Given an integer $n$ and starting/ending posts, $A$, $B$, or $C$, move $ n $ sorted rings from post $ A $ to post $ B $.}%
\begin{steps}
  \item If $n = 0$ or $from = to$:
    \begin{steps}
    \item return
    \end{steps}

  \item If \big[$(from = A)$ AND $(to = B)$\big] OR \\
    \nf{\quad}\big[$(from = B)$ AND $(to = C)$\big] OR \\
    \nf{\quad}\big[$(from = C)$ AND $(to = A)$\big]: \rcomm{One step}
      % Move n disks from 'from' to 'to' directly in one step of the cycle.
     \begin{steps}
      \item If \big[($from \neq A$) AND ($to \neq A$)\big]:
        \begin{steps}
        \item Let other := A
        \end{steps}
      \item Else if \big[($from \neq B$) AND ($to \neq B$)\big]:
        \begin{steps}
        \item Let other := B
        \end{steps}
      \item Else:
        \begin{steps}
        \item Let other := C
        \end{steps}
      \item \tmo{move-Cyclic}($n - 1$, $from$, $other$)
      \item Move the top ring from $from$ to $to$
      \item \tmo{move-Cyclic}($n - 1$, $other$, $to$)
      \end{steps}

      \item Else: \rcomm{Two steps}
        \begin{steps}
        \item If \(from = A\):
          \begin{steps}
          \item \(mid := B\)
          \end{steps}
        \item Else if \(from = B\):
          \begin{steps}
          \item \(mid := C\)
          \end{steps}
        \item Else:
          \begin{steps}
          \item \(mid := A\)
          \end{steps}

        \item \tmo{moveCyclic}($n - 1$, $from$, \ $mid$).

        \item Move the top ring from $from$ to $mid$

        \item Move the top ring from $mid$ to $to$

        \item \tmo{moveCyclic}($n - 1$, $mid$, $to$).
        \end{steps}
  \end{steps}
\end{quote}
\end{solution}
\pagebreak

\tbo{Exercise 2.1.3.} % PROBLEM 2.1.3
The \tit{double-cyclic towers of Hanoi} is like the cyclic towers of Hanoi problem except now the goal is to move the $n$ sorted rings from post $A$ to post $C$.

\begin{solution}\ % SOLUTION 2.1.3
  To call the function, use \tmo{double-cyclic-Hanoi}($n$, $A$, $C$, $B$).
  \begin{quote}
  \noindent\tul{def func \tmo{double-cyclic-Hanoi($n$, $from$, $to$, $spare$):}}%

  \doc{Given an integer $n$, a starting post, ending post, and spare post, move $ n $ sorted rings from the starting post $ from $ to the ending post $ to $.}%
  \begin{steps}
    \item If $(n = 0)$ OR $(from = to)$:
      \begin{steps}
        \item Return
      \end{steps}
    \item Let \(oneStep := \) \boolF~.
    \item If \big[($from = A$) AND ($ to = B $)\big] OR \\
      \nf{\quad}\big[($ from = B $) AND ($ to = C $)\big] OR \\
      \nf{\quad}\big[($from = C$) AND ($ to = A $)\big]: \rcomm{One step}
        \begin{steps}
        \item $\tmo{double-cyclic-Hanoi}(n - 1, from, spare, to)$.
        \item Move the top ring from \( from \) to \( to \).
        \item $\tmo{double-cyclic-Hanoi}(n - 1, spare, to, from)$.
        \end{steps}
    %   \end{steps}
    % \noindent\doc{test}
    % \begin{steps}
    \setcounter{stepsi}{3}
      \item Else: \rcomm{Two steps}
        \begin{steps}
        \item If \(from = A\):
          \begin{steps}
          \item \(next := B\)
          \end{steps}
        \item Else if \(from = B\):
          \begin{steps}
          \item \(next := C\)
          \end{steps}
        \item Else:
          \begin{steps}
          \item \(next := A\)
          \end{steps}
      \item \tmo{double-cyclic-Hanoi}($n - 1$, $from$, $next$, $ to $).
      \item Move the top ring from $ from $ to $ next $.
      \item Move the top ring from $ next $ to $ to $.
      \item \tmo{double-cyclic-Hanoi}($n - 1$, $next$, $to$, $from$).
      \end{steps}
    \end{steps}
  \end{quote}
\end{solution}
\pagebreak

\tbo{Exercise 2.1.4.} % PROBLEM 2.1.4
In the \tit{thick towers of Hanoi} problem, we have $ 3n $ rings of $ n $ distinct sizes with three copies of each ring. (Rings of the same size can stack on top of each other.) The goal is to move all $ 3n $ rings from post $ A $ to post $ B $.

\begin{solution}\ % SOLUTION 2.1.4
\begin{quote}
\noindent\tul{\tmo{def func thick-Hanoi($ n $, $ A $, $ B $, $ C $):}}%

\doc{Given an integer $n$, and posts $ A, B, $ and $ C $, move $ 3n $ rings from post $ A $ to post $ B $.}%

\begin{steps}
  \item If $ (n > 0) $:
  \begin{steps}
    \item \tmo{thick-Hanoi}($n-1$, $ A $, $C$, $ B $)

    \item For $i$ from 1 to 3:
    \begin{steps}
      \item Move the top ring from $ A $ to $ B $
    \end{steps}
    \item \tmo{thick-Hanoi}($n-1$, $ C $, $ B $, $ A $)
  \end{steps}
\end{steps}
\end{quote}
\end{solution}
\pagebreak

\tbo{Exercise 2.1.5.} % PROBLEM 2.1.5
In the \tit{triple towers of Hanoi} problem, we have $ 3n $ rings of $ n $ distinct sizes with three copies of each ring. The goal is to distribute the rings so that each post has $ n $ rings, one of each size, in order.

\begin{solution}\ % SOLUTION 2.1.5
\begin{quote}
\noindent\tul{def func \tmo{triple-Hanoi($n$, $A$, $B$, $C$):}}%

\doc{Given an integer $n$, a starting post, an ending post, and a spare post, distribute $ 3n $ rings so that each post has $ n $ rings, one of each size, in order.}%

\begin{steps}
  \item If $n = 1$:
    \begin{steps}
    \item Move the top ring from $ A $ to $ B $
    \item Move the top ring from $ A $ to $ C $
    \end{steps}
  % Base case: 3 rings of the same smallest size
  % The third copy remains on A, so now each peg has one of size 1
  \item Else:
  % 1) Distribute the top 3(n-1) (smaller) rings among the pegs
    \begin{steps}
    \item \tmo{triple-Hanoi}($n-1$, $A$, $B$, $C$) \lcomm{Distribute the smaller rings}
    \item Move the top ring from $ A $ to $ B $
    \item \tmo{triple-Hanoi}($n-1$, $C$, $B$, $A$) \lcomm{Free up C}
    \item Move the top ring from $ A $ to $ C $
    \item \tmo{triple-Hanoi}($n-1$, $B$, $A$, $C$) \lcomm{Place the smaller rings on top}
    \end{steps}
\end{steps}

\end{quote}
\end{solution}
\pagebreak

\tbo{Exercise 2.1.6.} % PROBLEM 2.1.6
In the \tit{American towers of Hanoi} problem, we have $ n $ rings with each ring colored red, white, or blue. The three posts are also colored red, white, and blue. Initially all the rings are stacked (in order of size) on the red post. The goal is to stack all the red rings on the red post, the blue rings on the blue post, and the white rings on the white post (again, in order of size).

\begin{solution}\ %
\begin{quote}
\noindent\tul{def func \tmo{American-Hanoi($n$, $start$):}}%

\doc{Given an integer $n$ and a starting post, distribute the $ n $ colored rings to the post of corresponding color such that each post is in order. Assume we have functions \tmo{getColor} and \tmo{getPost} such that \tmo{getColor}($k$) will return the color of the $ n-k $th largest ring and \tmo{getPost($color$)} will return the post of that color.}%

\begin{steps}
  \item If $ n > 0$:
  \begin{steps}
    \item Let $largestColor = getColor(n)$
    \item Let $target = postOfColor(largestRingColor)$
    \item If $(target = start)$ THEN
      %     // The largest ring n is already on its correct post.
      %     // We do NOT need to move ring n at all,
      %     // just solve for the top n-1 rings in the same way.
      \begin{steps}
      \item \tmo{American-Hanoi}($n-1$, $start$)
      \end{steps}
    \item Else:
    \begin{steps}
      %     // We must move ring n from 'startPost' to 'targetPost'.
      %     // First, move the smaller n-1 rings out of the way
      %     // onto the one remaining post (call it 'sparePost').
      \item Let $spare$ be the post that is neither $start$ nor $target$
    %     // Step 1: Move top n-1 rings from startPost -> sparePost
    %     //         (using the *classic* Hanoi method).
      \item \tmo{Towers-of-Hanoi}($ n-1 $, $ start $, $ spare $, $ target $)

    %     // Step 2: Now move the largest ring n onto its correct post.
    %     MOVE_RING(n, startPost, targetPost)
      \item Move the top ring from $ start $ to $ target $
    %     // Step 3: Finally, move the n-1 rings from 'sparePost'
    %     //         to *their* correct color posts by recursion.
      \item American-Hanoi($n-1$, $spare$)
      \end{steps}
  \end{steps}
\end{steps}
\end{quote}
\nf{} \\
\begin{quote}
\noindent\tul{def func \tmo{Towers-of-Hanoi($k$, $A$, $B$, $C$):}}%

\doc{Standard Towers of Hanoi helper function}%

\begin{steps}
  \item If $k > 0$:
    \begin{steps}
    \item \tmo{Towers-of-Hanoi}($k-1$, $A$, $C$, $B$)
    \item Move the top ring from $A$ to $B$
    \item \tmo{Towers-of-Hanoi}($k-1$, $C$, $B$, $A$)
    \end{steps}
\end{steps}
\end{quote}
\end{solution}
\pagebreak

\tbo{Exercise 2.2.3.} % PROBLEM 2.2.3
Given an undirected graph $ G = (V, E) $, return the maximum size of any matching.

(A \tit{matching} is a set of edges $M \sseq E$ with disjoint endpoints.)

\begin{solution}\ % SOLUTION 2.2.3
\begin{quote}%
\noindent\tul{def func \tmo{max-matching($G = (V, E)$):}}%

\doc{Given an undirected graph $ G = (V, E) $, return the maximum size of any matching.}%

\begin{steps}
  \item If $E = \emptyset$:
    \begin{steps}
    \item Return 0
    \end{steps}
  % \item Let $ G_{\nf{in}} := G $ and $ G_{\nf{ex}} := G $
  \item We know there is at least one edge $ e \in E $, say $ e := (u, v) $
  \item Let $ E_{\nf{temp}} := \{d\in E \mid u\in d \nf{ or } v\in d \nf{ and } d\neq e\} $ be the set of edges sharing a vertex with $ e $, but not including $ e $
  \item Let $ G_{\nf{in}} := (V\setminus \{u,v\},\ E\setminus E_{u,v})$ be the graph obtained by removing  $ u,v $ and all edges incident to either vertex (besides $ e $) from $ G $
  \item Let $ count_{\nf{in}} := 1+\tmo{max-matching}(G_{\nf{in}})$ \rcomm{Branch 1: Include $ e $}
  \item Let $ G_{\nf{ex}} := (V,\ E\setminus\{e\}) $ be the graph obtained by removing the edge $ e $ from $ G $
  \item Let $ count_{\nf{ex}} := \tmo{max-matching}(G_{\nf{ex}}) $ \rcomm{Branch 2: Exclude $ e $}
  \item Return $\max(count_{\nf{in}}, count_{\nf{ex}})$
\end{steps}
\end{quote}%
\end{solution}%
\pagebreak

\tbo{Exercise 2.2.4.} % PROBLEM 2.2.4
Given an undirected graph $ G = (V, E) $, return the maximum size of any independent set of vertices.

(An \tit{independent set} is a set of vertices $ S\sseq V $ such that no two vertices $ u,v\in S $ are connected by an edge.)

\begin{solution}\ % SOLUTION 2.2.4
\begin{quote}%
  \noindent\tul{def func \tmo{max-independent-set($G = (V, E)$):}}%

  \doc{Given an undirected graph $ G = (V, E) $, return the maximum size of any independent set of vertices.}%

  \begin{steps}
    \item If $ \abs{V} = 0 $:
      \begin{steps}
      \item Return 0
      \end{steps}
    \item Else if $ \abs{V} = 1 $:
      \begin{steps}
      \item Return 1
      \end{steps}
    \item For each vertex $ v \in V $:
      \begin{steps}
      \item Let $ G_{\nf{in}} $ be the graph obtained by removing all edges containing $ v $ and all other vertices contained in said edges from $ G $
      \item Let $ count_{\nf{in}} := 1+\tmo{max-independent-set}(G_{\nf{in}}) $ \rcomm{Branch 1: Include $ v $}
      \item Let $ G_{\nf{ex}} $ be the graph obtained by removing $ v $ from $ G $
      \item Let $ count_{\nf{ex}} := \tmo{max-independent-set}(G_{\nf{ex}}) $ \rcomm{Branch 2: Exclude $ v $}
      \item Return $\max(count_{\nf{in}}, count_{\nf{ex}})$
      \end{steps}
  \end{steps}
\end{quote}%
\end{solution}%
\pagebreak

\tbo{Exercise 2.2.5.} % PROBLEM 2.2.5
Given a sequence of numbers, $x_1, \ldots, x_n$, return the length of the longest (strictly) increasing subsequence of $x_1, \ldots, x_n$ over all subsequences that include $x_1$. (You may assume $n \geq 1$.)

(A \tit{subsequence} of $x_1, \ldots, x_n$ is a sequence of the form $x_{i_1}, x_{i_2}, \ldots, x_{i_k}$, where $ 1\leq i_1<i_2<\cdots<i_k\leq n $. A sequence of numbers $y_1, \ldots, y_\ell$ is strictly increasing if $ y_i < y_{i+1} $ for $ i = 1, \ldots, \ell-1 $.)

\begin{solution}\ % SOLUTION 2.2.5
\begin{quote}%
\noindent\tul{def func \tmo{longest-increasing-subsequence-including-first($x_1, \ldots, x_n$):}} \rcomm{LISIF for short}%

\doc{Given a sequence of numbers, $x_1, \ldots, x_n$, return the length of the longest strictly increasing subsequence of $x_1, \ldots, x_n$ over all subsequences that include $x_1$.}%

\begin{steps}
  \item Let $longest = 1$
  \item For each index $i$ from 2 to $n$:
    \begin{steps}
    \item If $x_i > x_1$:
      \begin{steps}
      \item Let $tail := \parof{x_{i+1}, x_{i+2}, \ldots, x_n}$
      \item Let $current := 1 + \tmo{LISIF}(tail)$
      \item Set $longest = \max(longest, current)$
      \end{steps}
    \end{steps}
    \item Return $longest$
  \end{steps}
\end{quote}%
\end{solution}%
\pagebreak

\tbo{Exercise 2.2.6.} % PROBLEM 2.2.6
Given a directed graph $G$ and two vertices $s, t \in V$, return true if $s$ can reach $t$ in $G$, and \boolF~otherwise.

($s$ can reach $t$ if either $s = t$ or there is a sequence of (directed) edges of the
form $(s, v_1), (v_1, v_2), (v_2, v_3), \ldots,$ $\ (v_{k-1}, v_k), (v_k, t)$. Note that the endpoint of one
edge is the initial point of the next edge. This is also called a (directed) walk in $G$ from $s$ to $t$.)

\begin{solution}\ % SOLUTION 2.2.6
\begin{quote}%
\noindent\tul{def func \tmo{reachable($G = (V, E)$, $ s $, $ t $):}}%

\doc{Given a directed graph $G$ and two vertices $s, t \in V$, return true if $s$ can reach $t$ in $G$, and \boolF~otherwise.}%

\begin{steps}
  \item If $s = t$:
    \begin{steps}
    \item Return \boolT~
    \end{steps}
  \end{steps}
  \doc{Random traversal through all paths accessible from vertex $ s $}
  \begin{steps}
  \setcounter{stepsi}{1}
  \item For each edge $e = (s, v) \in E$:
    \begin{steps}
    \item Return \tmo{reachable}$(v, t)$
    \end{steps}
  \item Return \boolF~
\end{steps}

\end{quote}%
\end{solution}%
\pagebreak

\tbo{Exercise 2.2.7.} % PROBLEM 2.2.7
Given $ n $ integers $ x_1,\ldots,x_n\in\Z $, returns \boolT~if there is a subset of indicies $ S\sseq [n] $ such that $ \sum_{i\in S}x_i = \sum_{i\in[n]\setminus S} x_i $, and \boolF~otherwise.

\begin{solution}\ % SOLUTION 2.2.7
\begin{quote}%
\noindent\tul{def func \tmo{partition($G = (V, E)$):}}%

\doc{Given $ n $ integers $ x_1,\ldots,x_n\in\Z $, returns \boolT~if there is a subset of indicies $ S\sseq [n] $ such that $ \sum_{i\in S}x_i = \sum_{i\in[n]\setminus S} x_i $, and \boolF~otherwise.}%

\begin{steps}
  \item Let $ total = \sum_{i\in [n]} x_i $
  \item If $ total $ is odd:
    \begin{steps}
    \item Return \boolF
    \end{steps}
  \item Else: \rcomm{ETS $ \exists S\sseq [n] $ such that $ \sum_{i\in S} x_i = \frac{total}{2} $}
    \begin{steps}
    \item Return \tmo{subset-sum}$(x_1,\ldots,x_n, \frac{total}{2})$
    \end{steps}

  \item IDK! I GIVE UP I AM SO TIRED SORRY
\end{steps}

\end{quote}%
\end{solution}%


\end{document}
