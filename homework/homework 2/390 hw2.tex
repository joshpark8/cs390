\documentclass{article}

\input{preamble}
\input{letterfont}
\input{macros}

\fancyhead[L]{\tbo{Josh Park \\ Prof. Quanrud}}
\fancyhead[C]{\tbo{CS 390ATA \\ Advanced Topics in Algorithms}}
\fancyhead[R]{\tbo{Spring 2025 \\ Homework 2}}

\begin{document}
\tbo{Exercise 5.1.} Below is a series of optimization problems that takes as input an array $A[1..n]$ of integers, and asks for optimal subsequences of $A$ satisfying certain properties. Design and analyze an algorithm for each of these problems, addressing items 1--5 from section 5.2.
% >>==========|| 5.1.2 ||==========<<
\setcounter{section}{5}
\setcounter{exercise}{1}
\setcounter{subexercise}{1}
\begin{subexercise}
  A sequence of numbers $ x_1,\ldots,x_k $ is \tit{convex} if $ x_{i+1}-x_i \geq x_i-x_{i-1} $ for $ i=2,\ldots,k-1 $. Compute the length of the longest convex subsequence of $ A $.
\end{subexercise}

\begin{proof}[Recursive spec.]
Let $ A[1..n] $ be a fixed array, and $ cache[1..n][1..n] $ be an $ n\times n $ matrix. Define the function \tmo{convex}$(i,j)$ to return the length of the longest convex subsequence ending in $A[j]\tand A[i]$, with $i>j$.
\end{proof}

\begin{proof}[Recursive implementation.]\
\begin{quote}
\defalgo{convex}{i,j}
\begin{steps}
  \item If $ j < 1 $ or $ i = 1 $, then return 2
  \item Else, return $ \max\limits_{1\leq k < j}
  \begin{cases}
    1 + \text{convex}(j, k) & \text{if } A[i] - A[j] \geq A[j] - A[k] \\
    2 & \text{else}
  \end{cases} $
\end{steps}
\end{quote}
\end{proof}

\begin{proof}[Dynamic Programming.]
 We can utilize dynamic programming to reduce running time by filling an $ \mcO(n^2) $ table $ cache[1..n][1..n] $, in which the entry $ cache[i][j] $ corresponds to $ \tmo{convex}(i,j) $.
\end{proof}

\begin{proof}[Usage.]
  To use this function, initialize some variable $ tmp := 1 $. Then iterate through the pairs $ (i,j) $ for which $ 1 \leq j < i < n $, and set $ tmp = \max\{tmp, \tmo{convex}(i,j)\} $. Finally, return $ tmp $.
\end{proof}

\begin{proof}[Analysis of running time.]
  The function \tmo{convex} fills an $ n\times n $ matrix without repeating any computations, which gives us $ \mcO(n\sq) $ subproblems.

  Each subproblem takes $ \mcO(n) $ time to complete in the worst case.

  Thus our final time complexity is $ \mcO(n\cb) $.
\end{proof}
\pagebreak

% >>==========|| 5.1.4 ||==========<<
\setcounter{subexercise}{3}
\begin{subexercise}
Compute both \begin{itemize}
\item the length of the longest increasing subsequences of $ A $ where the sum of integers is even,
\item the length of the longest increasing subsequences of $ A $ where the sum of integers is odd.
\end{itemize}
\end{subexercise}

\begin{note}
  Based on solution key
\end{note}

\begin{proof}[Recursive spec.]
Returns a 2-tuple ($ a,b $) where $ a\tand b $ represent the length of the LIS ending at $ i $ with sums of even and odd parity, respectively.
\end{proof}

\begin{proof}[Recursive implementation.]\
  \begin{quote}
  \defalgo{LIS-parity}{i}
  \begin{steps}
    \item If $ A[i] $ is even: set $ a := 1 $ and $ b := 0 $
    \item Else: set $ a := 0 $ and $ b := 1 $
    \item For $ 1 \leq j \leq i-1 $: \begin{steps}
      \item If $ A[j] < A[i] $: \begin{steps}
        \item Set ($ \hat a, \hat b $) := \tmo{LIS-parity}(j)
        \item If $ A[i] $ is even: set $ a := \max(a,\hat a + 1) $ and $ b := \max(b,\hat b+1) $
        \item Else: set $ a := \max(a,\hat b + 1) $ and $ b := \max(b,\hat a + 1) $
      \end{steps}
    \end{steps}
    \item return ($ a,b $)
  \end{steps}
  \end{quote}
\end{proof}

\begin{proof}[Dynamic Programming.]
  We can leverage dynamic programming by caching the result of the $ \mcO(n) $ subcalls.
\end{proof}

\begin{proof}[Usage.]
  return $ \max\limits_{i\in[n]} \tmo{LIS-parity}(i) $
\end{proof}

\begin{proof}[Analysis of running time.]
  The $ \mcO(n) $ subcalls each take $ \mcO(n) $ time, giving us a final time complexity of $ \mcO(n\sq) $.
\end{proof}
\pagebreak

% >>==========|| 5.1.5 ||==========<<
\begin{subexercise}
  Suppose each entry in A is also colored red, white, or blue. We say that a sequence is \tit{American} if the colors alternate red, white, blue, red, white, blue, \ldots.  The first number in the sequence can be any color. Compute the length of the longest increasing American subsequence of $ A $. (You can assume that you can look up the color of $A[i]$, for any $i \in [n]$, in constant time.)
\end{subexercise}

\begin{note}
  Based on solution key
\end{note}

\begin{proof}[Recursive spec.]
  Returns longest increasing American subsequence of $ A $ ending at $ i $.
\end{proof}

\begin{proof}[Recursive implementation.]\
Assumes usage of subroutine \tmo{patriotic}($ \alpha,\beta $) that returns true if color of $ A[\alpha] $ follows color of $ A[\beta] $ in sequence.
\begin{quote}
\defalgo{LIS-USA}{i}
\begin{steps}
  \item Define $ len := 1 $
  \item For $ 1\leq j < i $: \begin{steps}
    \item If $ A[j] < A[i] $ and \tmo{patriotic}($ i,j $): \begin{steps}
      \item Set $ len := \max\{len, \tmo{LIS-USA}(j)+1\} $
    \end{steps}
  \end{steps}
  \item return $ len $
\end{steps}
\end{quote}
\end{proof}

\begin{proof}[Dynamic Programming.]
  We leverage dynamic programming by caching the result of the $ \mcO(n) $ subcalls.
\end{proof}

\begin{proof}[Usage.]
  return $ \max\limits_{1\leq i\leq n}\{\tmo{LIS-USA}(i)\} $
\end{proof}

\begin{proof}[Analysis of running time.]
  Each subcall takes $ \mcO(n) $ time, so the final complexity is $ \mcO(n\sq) $.
\end{proof}
\pagebreak

% >>==========|| 5.1.6 ||==========<<
\begin{subexercise}
A sequence $ x_1,\ldots, x_k $ is a \tit{palindrome} if the reversed sequence is the same; i.e., $ (x_1,\ldots,x_k) = (x_k,\ldots,x_1) $. For example, \begin{align*}
  \tit{mom, dad, racecar,\tand gohangasalamiimalasagnahog}
\end{align*}
are all palindromes. Compute the length of the longest palindrome subsequence of $ A $.
\end{subexercise}

\begin{note}
  Based on solution key
\end{note}

\begin{proof}[Recursive spec.]
  Returns the length of the longest palindrome subsequence of $ A $ between indices $ i \tand j $.
\end{proof}

\begin{proof}[Recursive implementation.]\
\begin{quote}
\defalgo{pal}{i,j}
\begin{steps}
  \item If $ i = j $: return 1
  \item return $ \max\{ 2+\tmo{pal}(i+1,j-1),\ \tmo{pal}(i+1,j),\ \tmo{pal}(i,j-1) \} $
\end{steps}
\end{quote}
\end{proof}

\begin{proof}[Dynamic Programming.]
  We leverage dynamic programming by caching the result of the $ \mcO(n\sq) $ subcalls.
\end{proof}

\begin{proof}[Usage.]
  return $ \tmo{pal}(1,n) $
\end{proof}

\begin{proof}[Analysis of running time.]
  Each subcall takes $ \mcO(1) $ time, giving a final complexity of $ \mcO(n\sq) $
\end{proof}
\pagebreak

% >>===========|| 5.2 ||===========<<
\tbo{Exercise 5.2.} Let $ A[1..m] $ and $ B[1..n] $ be two arrays. Each of the following problems asks for some aspect (e.g., the length) of a common subsequence of $A[1..n] \tand B[1..n]$ satisfying or optimizing certain properties. (A common subsequence of $A \tand B$ is a sequence that is a subsequence of both $A \tand B$.) Design and analyze an algorithm for each of these problems, addressing items 1--5 from section 5.2.
\setcounter{exercise}{2}
\setcounter{subexercise}{0}

% >>==========|| 5.2.1 ||==========<<
\begin{subexercise} % 5.2.1
  Compute the length of the longest common subsequence of $A \tand B$.
\end{subexercise}

\begin{note}
  Based on solution key
\end{note}

\begin{proof}[Recursive spec.]
  Returns length of longest common subsequence between $ A[1..a] $ and $ B[1..b] $.
\end{proof}

\begin{proof}[Recursive implementation.]\
\begin{quote}
\defalgo{common}{a,b}
\begin{steps}
  \item If $ a=0 $ or $ b = 0 $: return 0
  \item If $ A[a] = B[b] $: return $ 1+\tmo{common}(a-1,b-1) $
  \item return $ \max \{\tmo{common}(a-1,b),\ \tmo{common}(a,b-1)\}$
\end{steps}
\end{quote}
\end{proof}

\begin{proof}[Dynamic Programming.]
Cache the solution to each of the $ \mcO(n\sq) $ subproblems
\end{proof}

\begin{proof}[Usage.]
return $ 2n - \tmo{common}(n,n) $
\end{proof}

\begin{proof}[Analysis of running time.]
Each of the $ \mcO(n\sq) $ subcalls takes $ \mcO(1) $ time, so the final comlexity is $ \mcO(n\sq) $
\end{proof}
\pagebreak

% >>==========|| 5.2.2 ||==========<<
\begin{subexercise} % 5.2.2
  Compute the length of the shortest common supersequence of $A \tand B$.
\end{subexercise}

\begin{note}
  Based on solution key
\end{note}

\begin{proof}[Recursive spec.]
  Returns the length of the shortest palindrome subsequence between $ A[a_1..a_2] $ and $ B[b_1..b_2] $
\end{proof}

\begin{proof}[Recursive implementation.]\
\begin{quote}
\defalgo{shortest-common-pal}{a_1,a_2,b_1,b_2}
\begin{steps}
  \item If $ a_1 > a_2 $ or $ b_1 > b_2 $: return 0
  \item If ($ a_1 = a_2 $ or $ b_1 = b_2 $) and $ A[a_1] = B[b_1] $: return 0
  \item If $ A[a_1] = A[a_2] $ and $ B[b_1] = B[b_2] $ and $ A[a_1] = B[b_1] $: return $ 2+\tmo{common-pal}(a_1+1, a_2-1, b_1+1, b_2-1) $
  \item return $\min\begin{cases}
    \tmo{shortest-common-pal}(a_1+1,a_2,b_1,b_2) \\
    \tmo{shortest-common-pal}(a_1,a_2-1,b_1,b_2) \\
    \tmo{shortest-common-pal}(a_1,a_2,b_1+1,b_2) \\
    \tmo{shortest-common-pal}(a_1,a_2,b_1,b_2-1)
  \end{cases}$
\end{steps}
\end{quote}
\end{proof}

\begin{proof}[Dynamic Programming.]
  Cache the solution to the $ n^4 $ subproblems
\end{proof}

\begin{proof}[Usage.]
  return $ \tmo{shortest-common-pal}(1,n,1,n) $
\end{proof}

\begin{proof}[Analysis of running time.]
  Each of the $ n^4 $ subproblems takes $ \mcO(1) $ time, so our final complexity is $ \mcO(n^4) $
\end{proof}
\pagebreak

% >>==========|| 5.2.3 ||==========<<
\begin{subexercise} % 5.2.3
  Compute the length of the longest common subsequence of $A \tand B$ that is a
  palindrome.
\end{subexercise}

\begin{note}
  Based on solution key
\end{note}

\begin{proof}[Recursive spec.]
  Returns the length of the longest palindrome subsequence between $ A[a_1..a_2] $ and $ B[b_1..b_2] $
\end{proof}

\begin{proof}[Recursive implementation.]\
\begin{quote}
\defalgo{longest-common-pal}{a_1,a_2,b_1,b_2}
\begin{steps}
  \item If $ a_1 > a_2 $ or $ b_1 > b_2 $: return 0
  \item If ($ a_1 = a_2 $ or $ b_1 = b_2 $) and $ A[a_1] = B[b_1] $: return 0
  \item If $ A[a_1] = A[a_2] $ and $ B[b_1] = B[b_2] $ and $ A[a_1] = B[b_1] $: return $ 2+\tmo{common-pal}(a_1+1, a_2-1, b_1+1, b_2-1) $
  \item return $\max\begin{cases}
    \tmo{longest-common-pal}(a_1+1,a_2,b_1,b_2) \\
    \tmo{longest-common-pal}(a_1,a_2-1,b_1,b_2) \\
    \tmo{longest-common-pal}(a_1,a_2,b_1+1,b_2) \\
    \tmo{longest-common-pal}(a_1,a_2,b_1,b_2-1)
  \end{cases}$
\end{steps}
\end{quote}
\end{proof}

\begin{proof}[Dynamic Programming.]
  Cache the solution to the $ n^4 $ subproblems
\end{proof}

\begin{proof}[Usage.]
  return $ \tmo{longest-common-pal}(1,n,1,n) $
\end{proof}

\begin{proof}[Analysis of running time.]
  Each of the $ n^4 $ subproblems takes $ \mcO(1) $ time, so our final complexity is $ \mcO(n^4) $
\end{proof}
\pagebreak

% >>==========|| 5.10 ||==========<<
\setcounter{exercise}{9}
\begin{exercise}
  ... [C]ompute the maximum number of sets that can be obtained
  in a game of Solitaire Set over the decks $A[1..m],\ B[1..n],\tand C[1..p]$. Design and analyze an algorithm for this problem.
\end{exercise}

\begin{proof}[Recursive spec.]
IDK!!
\end{proof}

\begin{proof}[Recursive implementation.]\
\begin{quote}
\defalgo{}{}
\begin{steps}
  \item .
\end{steps}
\end{quote}
\end{proof}

\begin{proof}[Dynamic Programming.]

\end{proof}

\begin{proof}[Usage.]

\end{proof}

\begin{proof}[Analysis of running time.]

\end{proof}

\begin{proof}[Proof of correctness.]

\end{proof}
\end{document}
