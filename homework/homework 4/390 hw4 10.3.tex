\documentclass{article}

\input{preamble}
\input{letterfont}
\input{macros}

\fancyhead[L]{\tbo{Josh Park, Amy Kang, Diya Singh \\ Prof. Kent Quanrud}}
\fancyhead[C]{\tbo{CS 390ATA \\Homework 4 (\theexercise)}}
\fancyhead[R]{\tbo{Spring 2025 \\ Page \thepage}}

\begin{document}
\tbo{Exercise 10.3.} After your glorious app PikPok hit number 1 in the app store, you're
preparing for version 2.
Obviously, it needs to be great.

You've gathered a list of $k$ features \llist{F}{1}{k} that you could potentially add to version 2.
However, there are complicated dependencies and requirements among them so you don't necessarily want to add all of them.
There are 3 types of specifications defined over pairs of features $F_i \tand F_j$: \begin{enumerate}
  \item Requirements: Your app must include either $F_i$ or $F_j$.
  \item Conflicts: You cannot include both $F_i$ and $F_j$.
  \item Dependencies: If you include $F_i$, then you must include $F_j$.
\end{enumerate}
Collectively, we call requirements, conflicts, and dependencies the \tit{feature specifications}.
The feature specifications are given in list form.
The high-level task is to decide which of the features to implement, based on the given feature specifications.
We have two versions of the problem. For each of the problems [below], either (a) design and analyze a polynomial time algorithm (the faster the better), or (b) prove that a polynomial time algorithm would imply a polynomial time algorithm for SAT.
\setcounter{section}{10}
\setcounter{exercise}{3}
% \setcounter{subexercise}{1}
\begin{subexercise}
In the idealistic feature selection problem, the task is to decide if there is a subset of features that satisfies all the feature specifications.
\end{subexercise}

\begin{solution}\ % SOLUTION 1.3.1
\begin{quote}%
\noindent\tun{def func \tmo{example($A[1..n]$):}}%

\doc{example spec}

\begin{steps}
  \item If $ n \leq 1 $, return $ 1 $.
  \item test \begin{steps}
      \item test
  \end{steps}
\end{steps}
\end{quote}
\end{solution}
\pagebreak

\begin{subexercise}
In the realistic feature selection problem, the task is to choose a subset of features that satisfies the maximum number of feature specifications
\end{subexercise}

\begin{solution}
solution
\end{solution}
\end{document}
