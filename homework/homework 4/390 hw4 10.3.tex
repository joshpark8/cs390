\documentclass{article}

\input{preamble}
\input{letterfont}
\input{macros}

\fancyhead[L]{\tbo{Josh Park, Amy Kang, Diya Singh \\ Prof. Kent Quanrud}}
\fancyhead[C]{\tbo{CS 390ATA \\Homework 4 (\theexercise)}}
\fancyhead[R]{\tbo{Spring 2025 \\ Page \thepage}}

\begin{document}
\tbo{Exercise 10.3.} After your glorious app PikPok hit number 1 in the app store, you're
preparing for version 2.
Obviously, it needs to be great.

You've gathered a list of $k$ features \llist{F}{1}{k} that you could potentially add to version 2.
However, there are complicated dependencies and requirements among them so you don't necessarily want to add all of them.
There are 3 types of specifications defined over pairs of features $F_i \tand F_j$: \begin{enumerate}
  \item Requirements: Your app must include either $F_i$ or $F_j$.
  \item Conflicts: You cannot include both $F_i$ and $F_j$.
  \item Dependencies: If you include $F_i$, then you must include $F_j$.
\end{enumerate}
Collectively, we call requirements, conflicts, and dependencies the \tit{feature specifications}.
The feature specifications are given in list form.
The high-level task is to decide which of the features to implement, based on the given feature specifications.
We have two versions of the problem. For each of the problems [below], either (a) design and analyze a polynomial time algorithm (the faster the better), or (b) prove that a polynomial time algorithm would imply a polynomial time algorithm for SAT.
\setcounter{section}{10}
\setcounter{exercise}{3}


% ------------------------------------ 10.3.1 SOLUTION -----------------------------------%

% \setcounter{subexercise}{1}
\begin{subexercise}
In the idealistic feature selection problem, the task is to decide if there is a subset of features that satisfies all the feature specifications.
\end{subexercise}

\begin{solution}
We claim that the case of idealistic feature selection reduces to a 2-SAT problem and therefore has a polynomial time solution.

For each feature $F_i$, define a boolean variable $f_i$ such that $f_i$ true \iff $F_i$ implemented. Then, notice that the constraints for each feature specification can be translated into the language of 2-SAT.
\begin{enumerate}
    \item [1.] Requirement: $(f_i \lor f_j)$
    \item [2.] Conflict: $(\bar{f_i} \lor \bar{f_j})$
    \item [3.] Dependency: $(\bar{f_i} \lor f_j)$
\end{enumerate}
Thus, the conjunction of all of the feature specifications as 2-variable clauses gives us a 2-SAT problem.
We can therefore determine satisfiability of the feature specifications in polynomial time by using the algorithm for 2-SAT.

\begin{subproof}[Correctness.]
Suppose 2-SAT returns true. Then there exists some assignment
$A: \{f_1, ..., f_k\} \rightarrow \{0,1\}$ satisfying the conjunction of all the feature-spec clauses.
Including each feature $F_i$ if and only if $A(f_i) = 1$ gives us a subset of features that satisfies all feature specifications.

On the other hand, if 2-SAT returns false, then there is no assignment on $\{f_1, ..., f_k\}$ that satisfies the conjunction of all the feature-spec clauses.
Hence, there is no subset of features that would satisfy all of the feature specifications.
\end{subproof}

\end{solution}
\pagebreak



% ------------------------------------ 10.3.2 SOLUTION -----------------------------------%

\begin{subexercise}
In the realistic feature selection problem, the task is to choose a subset of features that satisfies the maximum number of feature specifications
\end{subexercise}

\begin{solution}\
A realistic feature selection implies a polynomial time algorithm for SAT; we prove this creating a polynomial-time reduction from any Max 2-SAT problem to a realistic feature selection problem.

Consider a generic Max 2-SAT instance $f(x_1, ..., x_n)$ with $m$ clauses.

If we let $F_1, ..., F_n$ be features, where each $x_i$ is true if and only if its corresponding feature $F_i$ is chosen, then Max 2-SAT on $f(x_1, ..., x_n)$ directly corresponds to a realistic feature selection problem on $F_1, ..., F_n$. In particular, each clause constructed with variables $x_i$ and $x_j$ directly corresponds to a type of feature dependency:
\begin{enumerate}
    \item [1.] $(x_i \lor x_j)$: app must include either $F_i$ or $F_j$ (requirements).
    \item [2.] $(\bar{x}_i \lor \bar{x}_j)$: cannot include both $F_i$ and $F_j$ (conflicts).
    \item [3.] $(\bar{x}_i \lor x_j)$: if $F_i$ is included, $F_j$ must be included (dependencies).
    \item [4.] $(x_i \lor \bar{x}_j)$: if $F_j$ is included, $F_i$ must be included (dependencies).
\end{enumerate}

Hence, we can create a list of feature specifications on $F_1, ..., F_n$ from $f(x_1, ..., x_n)$ in polynomial time.

Therefore, if realistic feature selection on $F_1, ..., F_n$ has a polynomial-time solution, then so does Max 2-SAT on $f(x_1, ..., x_n)$


\begin{subproof}[Correctness.]
Let $S\sseq\{F_1,\ldots,F_n\}$ be a subset of features, and define `$F_i$ chosen' \iff `$x_i$ is true'. Then, the clauses $\{C_j\}_{j=1}^k$ are satisfied \iff $S$ meets the corresponding feature requirements.

Suppose the Max 2-SAT instance $f(x_1, ..., x_n)$ has at most $k$ satisfiable clauses, and label them $C_1, C_2, ..., C_k$.
Assume ad absurdum that $S$ satisfies more than $k$ feature specifications. Then, the corresponding 2-SAT instance would satisfy more than $k$ clauses, contradicting our assumption that the number of satisfiable clauses is bounded above by $k$. Hence, if no assignment satisfies more than $k$ clauses, there can not be some subset of features $S$ satisfying greater than than $k$ feature specifications.

Now, suppose the Max 2-SAT instance $f(x_1, ..., x_n)$ has greater than $k$ satisfiable clauses.
Assume ad absurdum that $S$ satisfies less than $k$ feature specifications.
Then, $f$ would necessarily satisfy less than $k$ clauses, which contradicts the assumption that $f$ has greater than $k$ satisfiable clauses.
Thus, there can not be some subset of features $S$ satisfying less than $k$ feature specifications.
\end{subproof}
\end{solution}
\end{document}
