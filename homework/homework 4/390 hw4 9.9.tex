\documentclass{article}

\input{preamble}
\input{letterfont}
\input{macros}

\fancyhead[L]{\tbo{Josh Park, Amy Kang, Diya Singh \\ Prof. Kent Quanrud}}
\fancyhead[C]{\tbo{CS 390ATA \\Homework 4 (\theexercise)}}
\fancyhead[R]{\tbo{Spring 2025 \\ Page \thepage}}

\begin{document}
\setcounter{section}{9}
\setcounter{exercise}{8}
\begin{exercise}
  Consider the following special case of SAT, which we will call \tit{k-occurrence-SAT} for a fixed parameter $k \in \N$.
  The input consists of a SAT formula $f (x_1, \ldots, x_n)$ in CNF such that every variable $x_i$ appears (as is, or negated) in at most $k$ clauses.
  The problem is to decide whether there is a satisfying assignment.
  For $k = 3$, either (a) design and analyze a polynomial time algorithm, or (b) show that a polynomial time algorithm for $k$-occurrence-SAT implies a polynomial time algorithm for (CNF-)SAT.\footnote{As a warmup, it might be helpful to first consider the case $k = 5$.
  If you figure out 5-occurrence SAT, but don't figure out 3-occurrence SAT, we will give partial credit for a solution to 5-occurrence SAT.}
\end{exercise}

\begin{solution} % SOLUTION 9.8
We claim that for $k=3$, the existance of a polynomial-time algorithm for $k$-occurrence-SAT implies a polynomial-time algorithm for SAT. To see this, we propose a polynomial-time reduction from any CNF-SAT formula $f(x_1, ..., x_n)$ to an corresponding 3-occurrence-SAT formula.

\begin{subproof}[Reduction.]
For each variable $x_i$ in $f$, let $\hat k$ be the number of occurrences of $x_i$ in $f$.

If $x_i$ has $\hat k \leq 3$ occurrences, then $f$ already meets the constraints of 3-occurrence-SAT.

In the case that there exists $x_i\in f$ with $\hat k > 3$ occurrences, we split $x_i$ into multiple variables.
We chose to split $x_i$ into $k$ equivalent variables $x_{i1} = x_{i2} =  ... = x_{ik}$ such that $j^{\text{th}}$ occurrence of $x_i$ in $f$ can be replaced by a new variable $x_{ij}$, $1 \leq j \leq \hat k$.

We also have to enforce equality between all the new variables $x_{ij}$.
This can be done by simply appending the clauses
\begin{align*}
    (\bar{x}_{i1} \lor x_{i2}) \ \land \ (\bar{x}_{i2} \lor x_{i3}) \ \land \ ... \ \land \ (\bar{x}_{i\hat k} \lor x_{i1})
\end{align*}
to the boolean formula. The addition of these $\hat k$ clauses adds 2 more occurrences of each $x_{ij}$, for a total of 3 occurrences.

When we perform the above substitution for all $x_i$ with more than 3 occurrences, notice that the size of the new formula, call it $f'$, is linearly proportional to the size of the original formula $f$, whence the reduction is polynomial-time.
\end{subproof}

% ------------------------------- CORRECTNESS ------------------------------- %


% \vspace{1cc}
\begin{subproof}[Correctness.]
Suppose we apply the reduction above to a CNF boolean formula $f$ to get a 3-occurrence boolean formula $f'$. We will prove that $f$ satisfiable \iff $f'$ satisfiable.

(\imp) If $f$ is satisfiable, then by definition there exists some SAT assignment $A: \{x_i\}_{i=0}^n \rightarrow \{0,1\}$ satisfying $f(x_1,..., x_n)$. By our construction above, the corresponding 3-occurrence formula $f'$ is also satisfied by assigning each ``duplicate'' variable $x_{ij}$ to the same value as the original variable $x_i$.

(\pmi) If $f'$ is satisfiable, then by definition there exists some SAT assignment $A: \{x_i\}_{i=0}^n \rightarrow \{0,1\}$ satisfying $f'$.
We also know if we have ``duplicate'' variables $x_{i1}, ... x_{ik}$ for some $x_i$, then $A(x_{i1}) = \ ... \ = A(x_{ik})$ is enforced by the equality clauses in $f'$. Hence, if we set the corresponding variable $x_i$ in $f$ to have the same assignment, $f$ is also satisfied.
\end{subproof}

\end{solution}
\pagebreak

\end{document}
