\documentclass{article}

\input{preamble}
\input{letterfont}
\input{macros}

\fancyhead[L]{\tbo{Josh Park, Amy Kang, Diya Singh \\ Prof. Kent Quanrud}}
\fancyhead[C]{\tbo{CS 390ATA \\ Homework 5 (\theexercise)}}
\fancyhead[R]{\tbo{Spring 2025 \\ Page \thepage}}

\begin{document}
\setcounter{section}{11}
\setcounter{exercise}{5}

\tbo{Exercise \theexercise} Let $ \llist{x}{1}{n}\in \N $.
For each of the following problems, either (a) design and analyze a polynomial time algorithm (the faster the better), or (b) prove that a polynomial time algorithm would imply a polynomial time algorithm for SAT.\textsuperscript{\hyperref[fn:sat]{2}} \\
\noindent\rule{2in}{0.4pt} \\
\parbox{\linewidth}{\small \textsuperscript{\label{fn:sat}2}You can use the solution of one subproblem to solve another, as long as there's no circular dependencies overall.}

%-------------------------------- 11.5.1 SOLUTION ---------------------------------%

\begin{subexercise}
  The \tit{partition problem} asks if one can partition $ \llist{x}{1}{n} $ into two parts such that the sums of each part are equal.
\end{subexercise}

\begin{solution}
We claim that a polynomial time solution for the partition problem would imply a polynomial time solution for SAT. To see this, we present a polynomial time reduction from subset sum, a problem known to be hard, to the partition problem.

\begin{note}[Notation]
Given some set $ S = \{\llist{s}{1}{n}\}$, we denote the sum $ \sum\limits_{s\in S}s $ with  $  \Sigma S $.
\end{note}

Consider an arbitrary instance of subset sum.
That is, suppose we have a set of positive integers \begin{align*}
  A := \{\llist{\alpha}{1}{n}\} \sseq \Z_+
\end{align*}
and a positive integer target value $ T\in \Z_+ $.
Now, let $ x := 2T-\Sigma A $, and define a new set $ \bar A := A\cup \{x\}$.
In the case that $ T < \Sigma A/{2} $, notice that the value of $ x $ will end up being negative.
However, if there exists some set $ B = \{\llist{\beta}{1}{k}\}\sseq A $ such that $ \Sigma B = T $ then $ A\setminus B $ sums to $ \Sigma A - T \leq \Sigma A/{2} $, so we can simply rephrase the problem to use $ (\Sigma A)- T$ as the target value.

Then, notice that \begin{align*}
  \Sigma \bar A &= \Sigma A + 2T-\Sigma A \\
  &= 2T.
\end{align*}
% Obviously, if there exists some valid partition of $ \bar A $ into two subsets with equal sums then each partitioned subset must sum to $ T $.
\begin{subproof}[Correctness.] (SS\imp PP)
  Suppose there exists some $ B\sseq A $ such that $ \Sigma B = T $.
  Consider the partition of $ \bar A $ defined as $ \bar B = B\cup \{x\} $.
  Then, \begin{align*}
    \Sigma \bar B= T + 2T-\Sigma A.
  \end{align*}
  The remaining partition is then $ C := \bar A\setminus \bar B $, and \begin{align*}
  \Sigma C &= 2T- T + 2T-\Sigma A \\
  &= T + 2T-\Sigma A,
  \end{align*}
  and we can see that these two sums are equal.
  Hence, the partition problem is solved.

  (SS\pmi PP)
  Suppose the set $ \bar A $ has a valid partition such that each of the two subsets sum to $ T $.
  Recall that $ \bar A $ is defined as the union of $ A $ and the singleton set $ \{x\} $.
  By the pigeonhole principle, we know that one of these subsets of $ \bar A $ is a subset of $ A $, whence the subset sum problem is solved.
\end{subproof}
Since each step in the reduction process takes only $ O(1)\tor O(n) $ time, the entire reduction can be done in polynomial time relative to the size of $ A $.
Thus, a polynomial time solution for the partition problem implies a polynomial time solution for SAT.
\end{solution}
\pagebreak

%-------------------------------- 11.5.2 SOLUTION ---------------------------------%

\begin{subexercise}
  The \tit{3-partition problem} asks if one can partition $ \llist{x}{1}{n} $ into 3 parts such that the sums of each part are all equal.
\end{subexercise}

\begin{solution}

\end{solution}
\pagebreak

%-------------------------------- 11.5.3 SOLUTION ---------------------------------%

\begin{subexercise}
  The \tit{any-k-partition problem} asks if one can partition $ \llist{x}{1}{n} $ into k parts, for any integer k $ \geq $ 2, such that the sums of each part are all equal.
\end{subexercise}

\begin{solution}

\end{solution}
\pagebreak

%-------------------------------- 11.5.4 SOLUTION ---------------------------------%

\begin{subexercise}
  The \tit{almost-partition problem} asks if one can partition $ \llist{x}{1}{n} $ into two parts such that the two sums of each part differ by at most 1.
\end{subexercise}

\begin{solution}

\end{solution}
\pagebreak

%-------------------------------- 11.5.5 SOLUTION ---------------------------------%

\begin{subexercise}
  \footnote[3]{IMO, this one is the trickiest.}Let $n$ be even.
  The \tit{perfect partition problem} asks if one can partition $ \llist{x}{1}{n} $ into two parts such that
  \begin{enumerate}[label=(\alph*)]
    \item Each part has the same sum.
    \item Each part contains the same number of $ x_i $'s.
  \end{enumerate}
\end{subexercise}

\begin{solution}

\end{solution}
\end{document}
