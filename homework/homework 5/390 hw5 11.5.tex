\documentclass{article}

\input{preamble}
\input{letterfont}
\input{macros}

\fancyhead[L]{\tbo{Josh Park, Amy Kang, Diya Singh \\ Prof. Kent Quanrud}}
\fancyhead[C]{\tbo{CS 390ATA \\ Homework 5 (\theexercise)}}
\fancyhead[R]{\tbo{Spring 2025 \\ Page \thepage}}

\begin{document}
\setcounter{section}{11}
\setcounter{exercise}{5}

\tbo{Exercise \theexercise} Let $ \llist{x}{1}{n}\in \N $.
For each of the following problems, either (a) design and analyze a polynomial time algorithm (the faster the better), or (b) prove that a polynomial time algorithm would imply a polynomial time algorithm for SAT.\textsuperscript{\hyperref[fn:sat]{2}} \\
\noindent\rule{2in}{0.4pt} \\
\parbox{\linewidth}{\small \textsuperscript{\label{fn:sat}2}You can use the solution of one subproblem to solve another, as long as there's no circular dependencies overall.}

%-------------------------------- 11.5.1 SOLUTION ---------------------------------%

\begin{subexercise}
  The \tit{partition problem} asks if one can partition $ \llist{x}{1}{n} $ into two parts such that the sums of each part are equal.
\end{subexercise}

\begin{solution}

\end{solution}
\pagebreak

%-------------------------------- 11.5.2 SOLUTION ---------------------------------%

\begin{subexercise}
  The \tit{3-partition problem} asks if one can partition $ \llist{x}{1}{n} $ into 3 parts such that the sums of each part are all equal.
\end{subexercise}

\begin{solution}

\end{solution}
\pagebreak

%-------------------------------- 11.5.3 SOLUTION ---------------------------------%

\begin{subexercise}
  The \tit{any-k-partition problem} asks if one can partition $ \llist{x}{1}{n} $ into k parts, for any integer k $ \geq $ 2, such that the sums of each part are all equal.
\end{subexercise}

\begin{solution}

\end{solution}
\pagebreak

%-------------------------------- 11.5.4 SOLUTION ---------------------------------%

\begin{subexercise}
  The \tit{almost-partition problem} asks if one can partition $ \llist{x}{1}{n} $ into two parts such that the two sums of each part differ by at most 1.
\end{subexercise}

\begin{solution}

\end{solution}
\pagebreak

%-------------------------------- 11.5.5 SOLUTION ---------------------------------%

\begin{subexercise}
  \footnote[3]{IMO, this one is the trickiest.}Let $n$ be even.
  The \tit{perfect partition problem} asks if one can partition $ \llist{x}{1}{n} $ into two parts such that
  \begin{enumerate}[label=(\alph*)]
    \item Each part has the same sum.
    \item Each part contains the same number of $ x_i $'s.
  \end{enumerate}
\end{subexercise}

\begin{solution}

\end{solution}
\end{document}
