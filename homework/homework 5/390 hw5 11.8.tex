\documentclass{article}

\input{preamble}
\input{letterfont}
\input{macros}

\fancyhead[L]{\tbo{Josh Park, Amy Kang, Diya Singh \\ Prof. Kent Quanrud}}
\fancyhead[C]{\tbo{CS 390ATA \\ Homework 5 (\theexercise)}}
\fancyhead[R]{\tbo{Spring 2025 \\ Page \thepage}}

\begin{document}
\setcounter{section}{11}
\setcounter{exercise}{8}

\tbo{Exercise \theexercise} (Approximating subset sum.) Let $ \eps\in (0, 1)$ be fixed.
Here we treat $ \eps $ as a fixed constant (like $ \eps $ = .1, for 10\% error); in particular, running times of the form $ O(n^{O(1/\eps)}) $ count as a polynomial.

A ($ 1 \pm \eps $)-approximation algorithm for subset sum is one that (correctly) either: \begin{enumerate}
  \item Returns a subset whose sum lies in the range $[( 1-\eps )T , ( 1+\eps )T ]$.
  \item Declares that there is no subset that sums to (exactly) $T$.
\end{enumerate}
Note that such an algorithm does not solve the (exact) subset sum problem.

%-------------------------------- 11.8.1 SOLUTION ---------------------------------%

\begin{subexercise}
  Suppose every input number $ x_i $ was ``small'', in the sense that $ x_i \leq \eps T $. Give a polynomial time ($ 1\pm \eps $)-approximation algorithm for this setting.
\end{subexercise}

\begin{solution}

\end{solution}
\pagebreak

%-------------------------------- 11.8.2 SOLUTION ---------------------------------%
\begin{subexercise}
  Suppose every input number $ x_i $ was ``big'', in the sense that $ x_i > \eps T $. Give a polynomial time ($ 1\pm \eps $)-approximation algorithm for this setting.
\end{subexercise}

\begin{solution}

\end{solution}
\pagebreak

%-------------------------------- 11.8.3 SOLUTION ---------------------------------%
\begin{subexercise}
  Now give a polynomial time ($ 1\pm \eps $)-approximation algorithm for subset sum in the general setting (with both big and small inputs).
\end{subexercise}

\begin{solution}

\end{solution}
\end{document}
