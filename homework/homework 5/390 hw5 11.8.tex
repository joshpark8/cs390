\documentclass{article}

\input{preamble}
\input{letterfont}
\input{macros}

\fancyhead[L]{\tbo{Josh Park, Amy Kang, Diya Singh \\ Prof. Kent Quanrud}}
\fancyhead[C]{\tbo{CS 390ATA \\ Homework 5 (\theexercise)}}
\fancyhead[R]{\tbo{Spring 2025 \\ Page \thepage}}

\begin{document}
\setcounter{section}{11}
\setcounter{exercise}{8}

\tbo{Exercise \theexercise} (Approximating subset sum.) Let $ \eps\in (0, 1)$ be fixed.
Here we treat $ \eps $ as a fixed constant (like $ \eps $ = .1, for 10\% error); in particular, running times of the form $ O(n^{O(1/\eps)}) $ count as a polynomial.

A ($ 1 \pm \eps $)-approximation algorithm for subset sum is one that (correctly) either: \begin{enumerate}
  \item Returns a subset whose sum lies in the range $[( 1-\eps )T , ( 1+\eps )T ]$.
  \item Declares that there is no subset that sums to (exactly) $T$.
\end{enumerate}
Note that such an algorithm does not solve the (exact) subset sum problem.

\begin{note}
  You may (and should) assume $ T,x_i\in\R_{\geq 0} $. The problem appears (computationally) hard otherwise.
\end{note}

%-------------------------------- 11.8.1 SOLUTION ---------------------------------%

\begin{subexercise}\label{qs:small}
  Suppose every input number $ x_i $ was ``small'', in the sense that $ x_i \leq \eps T $. Give a polynomial time ($ 1\pm \eps $)-approximation algorithm for this setting.
\end{subexercise}

\begin{solution}
Assuming $x_i \leq \epsilon T$ for all $i$, we can take a greedy approach that exploits the ``smallness'' of each $x_i$ to guarantee the construction of a subset-sum that does not overshoot the target range $[( 1-\eps )T , ( 1+\eps )T ]$.

% pseudocode --------- %
\begin{quote}%
\defalgo{approx-subset-sum}{\{x_1, ..., x_n\}, \epsilon, T}

\doc{returns a nonempty subset $S \subseteq \{x_1, ..., x_n\}$ such that $\sum S \in [( 1-\eps )T , ( 1+\eps )T ]$; returns the empty set if no such subset exists. }
\begin{steps}
    \item Let $S$ be an empty set.
    \item For $i$ from 1 to $n$ do
    \begin{steps}
        \item $S \leftarrow S \ \cup \{x_i\}$.
        \item If $\sum S \geq (1 - \epsilon) T$ then return $S$.
    \end{steps}
    \item Return $\emptyset$.
\end{steps}
\end{quote}
Note that since $T > 0$, $S \not= \emptyset$ is necessary when we successfully return a subset sum $S$; hence we distinguish between ``successful'' and ``unsuccessful'' outputs by whether the algorithm returns a nonempty or empty set.

\begin{subproof} [Runtime]
The algorithm above has asymptotic runtime in $O(n)$, which gives us a polynomial-time solution.
\end{subproof}

\vspace{-0.7cc}
\begin{subproof} [Correctness]
We prove correctness by showing (i) if we successfully return a nonempty subset $S$, then $\sum S$ is in the target range and (ii) if we do not successfully return a nonempty subset, it is indeed impossible for a subset of $\{x_1, ..., x_n\}$ to sum to the target range.
\begin{enumerate}
    \item [(i)] First, we assume that the algorithm successfully returns some nonempty subset $S = \{x_1, ..., x_i\}$ for some $i \ \in [1,n]$. Our return condition guarantees that $\sum S \geq (1 - \epsilon) T$; to complete the proof, we show that $\sum S \leq (1 + \epsilon) T$.

    We already know that $\sum S \smallsetminus \{x_i\} < (1 - \epsilon) T$, otherwise, we would have already returned $S = \{x_1, ..., x_{i-1}\}$ in a previous iteration. But $x_i \leq \epsilon T$, so
    \begin{align*}
        \sum S = x_i + \sum S \smallsetminus \{x_i\} \ \leq \epsilon \ T + (1 - \epsilon) T = T \leq (1 + \epsilon) T
    \end{align*}
    \vspace{-1.3cc}
    \item [(ii)] Now assume that we return an empty subset, indicating that the algorithm did not find a successful subset. Then for all $i$, $\sum_{j\ \in [1, i]} x_j < (1 - \epsilon) T$; in particular, $\sum_{j\ \in [1, n]} x_j < (1 - \epsilon) T$, where all $x_j > 0$. Clearly, this means that no subset of $\{x_1, ..., x_n\}$ will be large enough to reach the target range.
\end{enumerate}

\end{subproof}

\end{solution}
\pagebreak

%-------------------------------- 11.8.2 SOLUTION ---------------------------------%
\begin{subexercise} \label{qs:big}
  Suppose every input number $ x_i $ was ``big'', in the sense that $ x_i > \eps T $. Give a polynomial time ($ 1\pm \eps $)-approximation algorithm for this setting.
\end{subexercise}

\begin{solution}
  We can leverage the fact that if each $x_i > \eps T$, then any feasible subset summing to at most $T$ can contain \emph{at most} $\tfrac{1}{\eps}$ items.
  Hence, we can afford to enumerate all subsets of size up to $\frac{1}{\eps}$.

  \begin{quote}
  \defalgo{approx-subset-sum-big}{\{x_1, \dots, x_n\}, \epsilon, T}

  \doc{returns a subset $S$ whose sum is in $[(1-\eps)T,\,(1+\eps)T]$ if possible; otherwise returns $\emptyset$.}

  \begin{steps}
    \item For $k$ from 0 to $\lfloor 1/\eps \rfloor$ do
    \begin{steps}
      \item Enumerate all subsets of $\{x_1,\dots,x_n\}$ of size exactly $k$, and call each such subset $S_k$.
      \item If $\sum S_k \in [(1-\eps)T,\,(1+\eps)T]$, then return $S_k$.
    \end{steps}
    \item Return $\emptyset$.
  \end{steps}
  \end{quote}

  \begin{subproof}[Runtime]
  Since each $x_i > \eps T$, any subset of size $ >\frac{1}{\eps}$ would exceed $T$.
  Thus we can simply check all subsets of size at most $\frac{1}{\eps}$.
  The number of such subsets is
  \[
  \binom{n}{0} + \binom{n}{1} + \cdots + \binom{n}{\lfloor 1/\eps \rfloor},
  \]
  which is $O(n^{1/\eps})$.
  This is polynomial time for fixed $\eps$.
  \end{subproof}

  \begin{subproof}[Correctness]
    If there exists a subset summing to exactly $T$, then any such subset must have size at most $\frac{1}{\eps}$.
    We enumerate all such subsets such that if a feasible subset $S$ exists, we will find one with sum in $[(1-\eps)T,\,(1+\eps)T]$.
    Indeed, notice that $\sum S \leq T \leq (1+\eps)T$, and $\sum S \geq T \geq (1-\eps)T$ trivially if it equals $T$.

    If we return the empty set, then no combination of up to $\frac{1}{\eps}$ items falls within the target interval.
    In particular, no subset can sum exactly to $T$.

    Thus the procedure satisfies the $(1 \pm \eps)$-approximation requirement.
  \end{subproof}
\end{solution}
\pagebreak

%-------------------------------- 11.8.3 SOLUTION ---------------------------------%
\begin{subexercise}
  Now give a polynomial time ($ 1\pm \eps $)-approximation algorithm for subset sum in the general setting (with both big and small inputs).
\end{subexercise}

\begin{solution}
  We combine the ideas from \ref{qs:small} and \ref{qs:big}.
  Split the input into two sets:
  \begin{align*}
    \mcB = \{\,x_i \mid x_i > \eps T\}, \quad \mcS = \{\,x_i \mid x_i \le \eps T\}.
  \end{align*}

  \begin{quote}
  \defalgo{approx-subset-sum-general}{\{\llist{x}{1}{n}\}, \eps, T}

  \doc{returns a subset $S$ whose sum is in $[(1-\eps)T,\,(1+\eps)T]$ if possible; otherwise returns $\emptyset$.}
  \begin{steps}
    \item Enumerate all subsets of $\mcB$ of size up to $\lfloor 1/\eps \rfloor$.
    \item For each such subset $B \subseteq \mcB$:
    \begin{steps}
      \item Let $C \leftarrow T - \sum B$ be the remaining capacity for the small items.
      \item $ S_{\text{small}} \leftarrow $ \subcall{approx-subset-sum}{\mcS, \eps, C}
      \item If $\sum B + \sum S_{\text{small}} \in [(1-\eps)T,\,(1+\eps)T]$ \begin{steps}
        \item Return $B \cup S_{\text{small}}$.
      \end{steps}
    \end{steps}
    \item Return $\emptyset$.
  \end{steps}
  \end{quote}

  \begin{subproof}[Runtime]
  We only enumerate subsets of $\mcB$ up to size $\frac{1}{\eps}$, which is at most $O(n^{1/\eps})$.
  For each such subset, we run \subcall{approx-subset-sum}{\mcS, \eps, C} in $O(n)$.
  Thus for fixed $\eps$, the algorithm is polynomial in $n$ and $(1/\eps)$.
  \end{subproof}

  \begin{subproof}[Correctness]
  We consider two cases:

    \tit{Case 1:} Suppose there exists a subset
    \[
    S^* \subseteq \{x_1, \dots, x_n\} \quad \text{with} \quad \sum_{x \in S^*} x = T.
    \]
    Partition $ S^* $ into big and small items by letting
    \[
    B^* = S^* \cap \mcB \quad \text{and} \quad S^*_{\text{small}} = S^* \cap \mcS,
    \]
    so that $ S^* = B^* \cup S^*_{\text{small}} $.
    When the algorithm enumerates subsets of $ \mcB $, it will eventually consider a subset $ B $ corresponding to $ B^* $.
    Define the residual capacity
    \[
    C = T - \sum_{x \in B} x.
    \]
    Then, calling $ S_{\text{small}} = \subcall{approx-subset-sum}{\mcS, \eps, C} $ yields a subset of $ \mcS $ whose sum is within a factor of $ (1\pm\eps) $ of $ C $.
    Consequently, the combined sum $ \sum \lt[B \cup S_{\text{small}}\rt] $ falls within the interval $ [(1-\eps)T,\,(1+\eps)T] $.

    \tit{Case 2:} If no subset of the input sums to $ T $, then the algorithm returns $ \emptyset $.
    This outcome is correct, as it does not falsely claim the existence of a feasible subset.
  \end{subproof}
\end{solution}
\pagebreak

\end{document}
