\documentclass{article}
\setlength{\headheight}{22.50113pt}
\addtolength{\topmargin}{-10.50113pt}

\input{preamble}
\input{letterfont}
\input{macros}

\fancyhead[L]{\tbo{Josh Park, Amy Kang, Diya Singh \\ Prof. Kent Quanrud}}
\fancyhead[C]{\tbo{CS 390ATA\\Homework 5 (\theexercise)}}
\fancyhead[R]{\tbo{Spring 2025 \\ Page \thepage}}

\begin{document}
\setcounter{section}{12}
\setcounter{exercise}{10}

\tbo{Exercise \theexercise} Let $ G = (V, E) $ be a directed graph with $ m $ edges and $ n $ vertices, where each vertex $v \in V$ is given an integer label $\ell(v) \in\N$.
The goal is to find the length of the longest path\textsuperscript{\hyperref[fn:path]{3}} in $ G $ where the labels of the vertices are (strictly) increasing.\\
\noindent\rule{2in}{0.4pt} \\
\parbox{\linewidth}{\small \textsuperscript{\label{fn:path}3}Recall that a path is a walk that does not repeat vertices.}

%-------------------------------- 12.10.1 SOLUTION ---------------------------------%
\begin{subexercise}
  Suppose $G$ is a DAG. For this problem, either (a) design and analyze a polynomial time algorithm (the faster the better), or (b) prove that a polynomial time algorithm would imply a polynomial time algorithm for SAT.
\end{subexercise}

\begin{solution}
We claim the following polynomial-time algorithm suffices.

\begin{quote}%
\defalgo{lip-dag}{s}

\doc{given $s \in V$ in a DAG, computes the length of the longest strictly-increasing path starting at vertex $s$, measured by the number of edges.}

\begin{steps}
  \item Let $m \leftarrow 0$.
  \item For $(s,v)\in \delta^+(s)$ such that $\ell(v) > \ell(s)$ do
    \begin{steps}
        \item m $\leftarrow \max \ (m, 1 + \subcall{lip}{v})$
    \end{steps}
  \item return m
\end{steps}
\end{quote}

We can solve the original problem by computing \(\max_{ \ \forall v\in V}\subcall{lip-dag}{v}\).

\begin{subproof}[Runtime]
If we cache the return value of \subcall{lip-dag}{v} for all $v \in V$ (as well as the maximum of all these return values), our algorithm has the following runtime complexity:
\begin{align*}
    O \left( n \cdot \left( 1+\sum_{v\in V} d^+(v) \right) \right) = O \left( m+n \right)
\end{align*}

% Now, solving the original problem by calling \subcall{lip-dag}{v} for all $v\in V$ will also be \(O(m+n)\) since each call is cached, and so our desired m values can be calculated without recomputation for some $v\in V$ that has already been calculated.%

\end{subproof}

\begin{subproof}[Correctness]
The correctness of this algorithm follows from performing induction, in reverse topological order, according to the recursive specification. We claim that for all vertices $s \in V$, the algorithm correctly computes the length of the longest strictly increasing path starting with vertex $s$, measured in the number of edges.

In the base case, $s$ is the last vertex in topological order, so $s$ is a sink and the longest increasing path starting from $s$ has length 0, as returned in the algorithm.

Now assume that the claim holds for some vertex $s$ as well as all vertices that follow $s$ in topological order. Take $s' \in V$ to be a vertex that immediately preceeds $s$ in topological order, and notice that the claim holds for all $v$ such that $(s',v) \in \delta^{+}(s')$.
If $s'$ has no out-neighbors with larger weight, then the algorithm correctly returns a maximum increasing path length of 0. Otherwise, the longest path starting at $s'$ has length
\begin{align*}
    \max_{v \ : \ (s',v) \ \in \ \delta^{+}(s'), \ell(v) > \ell(s')} \ \subcall{lip-dag}{v}
\end{align*}
as computed in the algorithm.

Hence, by induction, the claim---algorithmic correctness---holds for all $s \in V$.

\end{subproof}

\end{solution}
\pagebreak





%-------------------------------- 12.10.2 SOLUTION ---------------------------------%
\begin{subexercise}
  Consider now the problem for general graphs. For this problem, either (a) design and analyze a polynomial time algorithm (the faster the better), or (b) prove that a polynomial time algorithm would imply a polynomial time algorithm for SAT.
\end{subexercise}

\begin{solution}
The intuition here is that, because it is not possible to have a strictly-increasing cycles, we can treat $G$ just as we would a DAG.

We claim the same polynomial-time algorithm as in 12.10.1 solves this problem.

\begin{quote}%
\defalgo{lip}{s}

\doc{given $s \in V$, computes the length of the longest strictly-increasing path starting at vertex $s$, measured by the number of edges.}

\begin{steps}
  \item Let $m \leftarrow 0$.
  \item For $(s,v)\in \delta^+(s)$ such that $\ell(v) > \ell(s)$ do
    \begin{steps}
        \item m $\leftarrow \max \ (m, 1 + \subcall{lip}{v})$
    \end{steps}
  \item return m
\end{steps}
\end{quote}

Again, we can solve the original problem by computing \(\max_{ \ \forall v\in V}\subcall{lip}{v}\).

\begin{subproof}[Runtime]
If we cache the return value of \subcall{lip}{v} for all $v \in V$ (as well as the maximum of all these return values), our algorithm has the following runtime complexity:
\begin{align*}
    O \left( n \cdot \left( 1+\sum_{v\in V} d^+(v) \right) \right) = O \left( m+n \right)
\end{align*}
\end{subproof}

\begin{subproof}[Correctness]
To clarify the DAG-like nature of this traversal (i.e. why there are no cyclic dependencies in the recursive calls), consider calling \texttt{lip} on some vertex $s \in V$. We make recursive calls to adjacent vertices $v \in V$, $(s,v) \in \delta^{+}(s)$ only when $\ell(v) > \ell(s)$, which creates a sort of topological ordering-by-weight on $G$.

Explicitly: we claim that for all vertices $s \in V$, the algorithm correctly computes the length of the longest strictly increasing path starting with vertex $s$, measured in the number of edges. To prove this, we perform induction based on the weight $\ell(v)$ of all $v \in V$.

In the base case, consider $s \in V$ with maximal weight; that is, $\ell(s) \geq \ell(v)$ for all $v \in V$. Then since $\ell(v) > \ell(s)$ is always false, \texttt{lip}$(s)$ makes no recursive calls, and the algorithm returns 0, as desired.

Now assume that the claim holds for some vertex $s$ as well as all vertices $v \in V$ such that $\ell(v) \geq \ell(s)$.

Take $s' \in V$ to be a vertex that immediately preceeds $s$ in weight order (that is, $\ell(s') \leq \ell(s)$ and there exists no $v \in V$ for which $\ell(s') < \ell(v) < \ell(s)$), and notice that the claim holds for all $v$ such that $(s',v) \in \delta^{+}(s')$ and $\ell(v) > \ell(s')$ by assumption.

If $s'$ has no out-neighbors with larger weight, then the algorithm correctly returns a maximum increasing path length of 0. Otherwise, the longest path starting at $s'$ has length
\begin{align*}
    \max_{v \ : \ (s',v) \ \in \ \delta^{+}(s'), \ \ell(v) > \ell(s')}  \ \subcall{lip}{v}
\end{align*}
as computed in the algorithm.

By induction, we can conclude that \texttt{lip}$(s)$ returns the length of the longest increasing path for all $s \in V$.
\end{subproof}
\end{solution}
\pagebreak






%-------------------------------- 12.10.3 SOLUTION ---------------------------------%
\begin{subexercise}
  Suppose instead we ask for the length of the longest path in $G$ where $G$ is a general graph and the labels of the vertices are weakly increasing.\footnote[4]{A sequence $ \llist{x}{1}{k} $ is weakly increasing if $ x_1\leq x_2\leq \cdots\leq x_k $.}
  For this problem, either (a) design and analyze a polynomial time algorithm (the faster the better), or (b) prove that a polynomial time algorithm would imply a polynomial time algorithm for SAT.
\end{subexercise}

\begin{solution}
We claim that a polynomial time solution for the partition problem would imply a polynomial time solution for SAT. To see this, we present a polynomial time reduction from the longest path problem, which is known to be hard, to the longest weakly increasing path problem.

The approach is simple: given a directed graph $G$, we label every vertex with the same weight $1$ to get a vertex-weighted graph $G'$. This is clearly a polynomial-time reduction.

\begin{subproof} [Correctness]
We claim that \texttt{longest-weakly-increasing-path}($G'$) computes \texttt{longest-path}($G$).

If we take any path $P$ in $G$, then the exact same path in $G'$ is weakly increasing, since $1 \leq 1 \leq 1 \leq ...$. Conversely, take any weakly-increasing path $P'$ in $G'$. Since removing vertex weights does not alter reachability, $P'$ is also a path in $G$.

This correspondence preserves path lengths; hence, the maximum path length \texttt{longest-path}($G$) is exactly the maximum weakly-increasing path \texttt{longest-weakly-increasing-path}($G'$).
\end{subproof}
\end{solution}
\pagebreak

\end{document}
