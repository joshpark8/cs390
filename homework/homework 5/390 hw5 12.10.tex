\documentclass{article}
\setlength{\headheight}{22.50113pt}
\addtolength{\topmargin}{-10.50113pt}

\input{preamble}
\input{letterfont}
\input{macros}

\fancyhead[L]{\tbo{Josh Park, Amy Kang, Diya Singh \\ Prof. Kent Quanrud}}
\fancyhead[C]{\tbo{CS 390ATA\\Homework 5 (\theexercise)}}
\fancyhead[R]{\tbo{Spring 2025 \\ Page \thepage}}

\begin{document}
\setcounter{section}{12}
\setcounter{exercise}{10}

\tbo{Exercise \theexercise} Let $ G = (V, E) $ be a directed graph with $ m $ edges and $ n $ vertices, where each vertex $v \in V$ is given an integer label $\ell(v) \in\N$.
The goal is to find the length of the longest path\textsuperscript{\hyperref[fn:path]{3}} in $ G $ where the labels of the vertices are (strictly) increasing.\\
\noindent\rule{2in}{0.4pt} \\
\parbox{\linewidth}{\small \textsuperscript{\label{fn:path}3}Recall that a path is a walk that does not repeat vertices.}

%-------------------------------- 12.10.1 SOLUTION ---------------------------------%
\begin{subexercise}
  Suppose $G$ is a DAG. For this problem, either (a) design and analyze a polynomial time algorithm (the faster the better), or (b) prove that a polynomial time algorithm would imply a polynomial time algorithm for SAT.
\end{subexercise}

\begin{solution}

\end{solution}
\pagebreak

%-------------------------------- 12.10.2 SOLUTION ---------------------------------%
\begin{subexercise}
  Consider now the problem for general graphs. For this problem, either (a) design and analyze a polynomial time algorithm (the faster the better), or (b) prove that a polynomial time algorithm would imply a polynomial time algorithm for SAT.
\end{subexercise}

\begin{solution}

\end{solution}
\pagebreak

%-------------------------------- 12.10.3 SOLUTION ---------------------------------%
\begin{subexercise}
  Suppose instead we ask for the length of the longest path in $G$ where $G$ is a general graph and the labels of the vertices are weakly increasing.\footnote[4]{A sequence $ \llist{x}{1}{k} $ is weakly increasing if $ x_1\leq x_2\leq \cdots\leq x_k $.}
  For this problem, either (a) design and analyze a polynomial time algorithm (the faster the better), or (b) prove that a polynomial time algorithm would imply a polynomial time algorithm for SAT.
\end{subexercise}

\begin{solution}

\end{solution}
\end{document}
