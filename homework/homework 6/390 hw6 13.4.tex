\documentclass{article}

\input{preamble}
\input{letterfont}
\input{macros}

\fancyhead[L]{\tbo{Josh Park, Amy Kang, Diya Singh \\ Prof. Kent Quanrud}}
\fancyhead[C]{\tbo{CS 390ATA \\ Homework 6 (\theexercise)}}
\fancyhead[R]{\tbo{Spring 2025 \\ Page \thepage}}

\begin{document}
\setcounter{section}{13}
\setcounter{exercise}{3}
% >>==========|| 13.4 ||==========<<
\begin{exercise}
Let $ G=(V,E) $ be a directed graph.
We say that a set of vertices is \tit{almost independent} if each $ v\in S $ has at most one neighbor in $ S $.\footnote[5]{Two vertices $u \tand v$ are neighbors if they are connected by an edge.}
Consider the problem of computing the maximum cardinality of any almost independent set of vertices.
For this problem, either (a) design and analyze a polynomial time algorithm (the faster the better), or (b) prove that a polynomial time algorithm would imply a polynomial time algorithm for SAT.
\end{exercise}

\begin{solution}
  We claim that the maximum cardinality almost independent set (MCAIS) problem is NP-complete.
  To see this, we present a polynomial time reduction from the maximum cardinality independent set (MCIS) problem, a problem we saw to be NP-complete in class, to MCAIS.

  \begin{note}
    Since the condition for vertices $ v,w\in V $ to be neighbors only requires that there be a connection between the two, the directional property of the edges in $ E $ can be ignored.
    Hence, we will not discriminate between two edges $ (v,w) $ and $ (w,v) $.
  \end{note}

  Consider an arbitrary instance of independent set.
  That is, suppose we have a graph $ G=(V,E) $.
  The MCAIS problem seeks the largest $ S\ss V $ such that $ (v,w)\not\in E $ for all $ v,w\in S $.

  Create an auxiliary graph $ G'=(V',E') $, where $ V'=\{v_i'\st v_i\in V\}\cup V $ and $ E'=\{(v_i,v_i')\st v_i\in V\}\cup E $.
  Then, $ V' $ contains duplicates of each vertex in $ V $ and $ E' $ contains each edge in $ E $, as well as an additional edge connecting each pair of vertices $ v_j $ and $ v_j' $.

  \begin{subproof}[Correctness.]
    (MCIS\imp MCAIS)
    Suppose we have a set $ S\ss V $ that solves MCIS for $ G $.
    Create a new set $ S'=\{v_i'\st v_i\in S\}\cup S $.
  \end{subproof}
\end{solution}
\pagebreak

\end{document}
