\documentclass{article}

\input{preamble}
\input{letterfont}
\input{macros}

\fancyhead[L]{\tbo{Josh Park, Amy Kang, Diya Singh \\ Prof. Kent Quanrud}}
\fancyhead[C]{\tbo{CS 390ATA \\ Homework 6 (\theexercise)}}
\fancyhead[R]{\tbo{Spring 2025 \\ Page \thepage}}

\begin{document}
\setcounter{section}{13}
\setcounter{exercise}{6}
\tbo{Exercise \theexercise.} Recall the dominating set problem\textsuperscript{\hyperref[fn:dmset]{$\dagger$}} from section 13.3.
  Here we will consider the weighted version where the vertices are given positive weights, and the goal is to compute the minimum weight dominating set.
  For each of the following problems, either (a) design and analyze a polynomial time algorithm (the faster the better), or (b) prove that a polynomial time algorithm would imply a polynomial time algorithm for SAT. \\
\noindent\rule{2in}{0.4pt} \\
\parbox{\linewidth}{\small \textsuperscript{\label{fn:dmset}$\dagger$}A set of vertices $S \sseq V$ is a \tit{dominating set} if every vertex $v \in V$ is either in $S$ or the neighbor of a vertex in $S$.
  The minimum dominating set problem is to compute the minimum cardinality dominating set.}

% >>==========|| 13.6 ||==========<<
\begin{subexercise}
  The minimum weight dominating set problem for intervals, \tit{with the additional assumption that} no two intervals are nested.
  To state it more precisely: the input consists of $ n $ weighted intervals $ \mcI $.
  The non-nested assumptions means that for any two intervals $ I, J \in \mcI $, we never have $ I $ contained in $ J $ or $ J $ contained in $ I $.

  The goal is to compute the minimum weight subset $ S \sseq \mcI $ of intervals such that every interval in $ \mcI $ is either in $ S $ or overlaps some interval in $ S $.
\begin{itemize}
  \item (For 1 pt. extra credit) Extend your algorithm to general intervals.\footnote[8]{Of course, anyone who has already solved the general case automatically solves the special case where no two intervals are nested.}
\end{itemize}
\end{subexercise}

\begin{solution}

\end{solution}
\pagebreak

\begin{subexercise}
  The minimum weight dominating set problem in trees.
\end{subexercise}

\begin{solution}

\end{solution}
\pagebreak
\end{document}
